\begin{center}
\Large

\textbf{Obrzędy ślubne}\\[0.2cm] 

\normalsize
\end{center}

\textcolor{red}{Kapłan, uklęknąwszy na najniższym stopniu ołtarza, intonuje hymn, a zgromadzeni śpiewają go dalej. Po odśpiewaniu pierwszej zwrotki wszyscy wstają:}\\[0.3cm]

\begin{Parallel}[v]{0.485\textwidth}{0.485\textwidth}
\ParallelLText{
\quad Veni, Creátor, Spíritus,

Mentes tuórum vísita,

Imple supérna grátia,

Quæ tu creásti, péctora.

\quad Qui díceris Paráclitus,

Altíssimi donum Dei,

Fons vivus, ignis, cáritas

Et spiritális únctio.

\quad Tu septifórmis múnere,

Dígitus patérnæ déxteræ,

Tu rite promíssum Patris,

Sermóne ditans gúttura.

\quad Accénde lumen sénsibus,

Infúnde amórem córdibus,

Infírma nostri córporis

Virtúte firmans pérpeti.

\quad Hostem repéllas lóngius

Pacémque dones prótinus:

Ductóre sic te prǽvio

Vitémus omne nóxium.

\quad Per te sciámus da Patrem,

Noscámus atque Fílium

Teque utriúsque Spíritum

Credámus omni témpore.

\quad Deo Patri sit glória

Et Fílio, qui a mórtuis

Surréxit, ac Paráclito

In sæculórum sǽcula. Amen.
}

\ParallelRText{
\quad O Stworzycielu Duchu, przyjdź,

Nawiedź dusz wiernych Tobie krąg,

Niebieską łaskę zesłać racz

Sercom, co dziełem są Twych rąk.

\quad Pocieszycielem jesteś zwan

I Najwyższego Boga dar,

Tyś namaszczenie naszych dusz,

Zdrój żywy, miłość, ognia żar.

\quad Ty darzysz łaską siedemkroć,

Bo moc z prawicy Ojca masz,

Przez Ojca obiecany nam,

Mową wzbogacasz język nasz.

\quad Światłem rozjaśnij naszą myśl,

W serca nam miłość świętą wlej

I wątłą słabość naszych ciał

Pokrzep stałością mocy swej.

\quad Nieprzyjaciela odpędź w dal

I Twym pokojem obdarz wraz.

Niech w drodze za przewodem Twym

Miniemy zło, co kusi nas.

\quad Daj nam przez Ciebie Ojca znać,

Daj, by i Syn poznany był,

I Ciebie, jedno Tchnienie Dwóch,

Niech wyznajemy z wszystkich sił.

\quad Niech Bogu Ojcu chwała brzmi,

Synowi, który zmartwychwstał

I Temu, co pociesza nas,

Niech hołd wieczystych płynie chwał. 

Amen.
}
\end{Parallel}

\begin{center}
\textcolor{red}{Po zakończeniu hymnu kapłan śpiewa:}\\[0.2cm]
\end{center}

\begin{Parallel}[v]{0.485\textwidth}{0.485\textwidth}
\ParallelLText{
\versicle Emítte Spíritum tuum et creabúntur.

\response Et renovábis fáciem terræ.

Orémus.

Deus, qui corda fidélium Sancti Spíritus illustratióne docuísti: \SmallCross da nobis in eódem Spíritu recta sápere; et de eius semper consolatióne gaudére. Per Christum, Dómi-num nostrum.

\response Amen.
}
\ParallelRText{
\versicle Ześlij Ducha Twego, a powstanie życie.

\response I odnowisz oblicze ziemi.

Módlmy się.

Boże, któryś pouczył serca wiernych światłem Ducha Świętego, \SmallCross daj nam w tymże Duchu poznać, co jest prawe,  i Jego pociechą zawsze się radować. Przez Chrystusa, Pana naszego.

\response Amen.
}
\end{Parallel}

\begin{center}
\large
\textbf{Poświęcenie obrączek}
\end{center}

\begin{center}
\textcolor{red}{Kapłan poświęca obrączki przyniesione przez ministranta słowami:}\\[0.2cm]
\end{center}

\begin{Parallel}[v]{0.485\textwidth}{0.485\textwidth}
\ParallelLText{
\versicle Adiutórium nostrum \SmallCross in nómine Dómini.

\response Qui fecit cælum et terram.

\versicle Dómine, exáudi oratiónem meam.

\response Et clamor meus ad te véniat.\\

\versicle Dóminus vobíscum.

\response Et cum spíritu tuo.

Orémus.

Creátor et conservátor géneris humáni, dator grátiæ spirituális, cónditor salútis ætérnæ: quǽsumus, bene \SmallCross dícere dignéris hos ánulos, quos nos in nómine tuo sancto benedícimus; * ut qui eos portáverint, in tua pace consístant, in tua voluntáte permáneant, et in tuo amóre vivant, crescant et senéscant atque multiplicéntur in longitúdinem diérum. Per Christum, Dóminum nostrum.\\ \\

\response Amen.
}
\ParallelRText{
\versicle Wspomożenie nasze \SmallCross w imieniu Pana.

\response Który stworzył niebo i ziemię.

\versicle Panie, wysłuchaj modlitwę moją.

\response A wołanie moje niech do Ciebie przyjdzie.

\versicle Pan z wami.

\response I z duchem twoim.

Módlmy się.

Stwórco i Zachowawco rodzaju ludzkiego, Dawco łaski duchowej i Sprawco zbawienia wiecznego, prosimy Cię: racz pobłogo \SmallCross sławić te obrączki, które błogosławimy w święte imię Twoje. Niech ci, którzy będą je nosić, trwają w Twym pokoju, zgadzają się z Twoją wolą, żyją i wzrastają w Twojej miłości i niech się doczekają podeszłego wieku, a poprzez liczne potomstwo niech sięgają w daleką przyszłość. Przez Chrystusa, Pana naszego.

\response Amen.
}
\end{Parallel}

\begin{center}
\textcolor{red}{Kapłan kropi obrączki wodą święconą. Ministrant odkłada potem tackę z poświęconymi obrączkami na ołtarz. Następnie narzeczeni wstępują do ołtarza i pozostają na jego najwyższym stopniu. Kapłan zaś, stojąc w pośrodku podnóżka ołtarza, przechodzi do Skrutynium.}

\textcolor{red}{	\tikzset{
		pgfornamentstyle/.style={scale=.25}
	}
	\foreach \i in {70} {\expandafter\pgfornament\expandafter{\i}\ }
	\textbf{PRZYSIĘGA MAŁŻEŃSKA} 
	\tikzset{
	pgfornamentstyle/.style={scale=.25}
}
\foreach \i in {70} {\expandafter\pgfornament\expandafter{\i}\ }
}

\textcolor{red}{Po wypowiedzeniu przysięgi małżeńskiej i nałożeniu obrączek zgromadzeni, po zaintonowaniu przez kantora albo organistę, śpiewają:
}
\end{center}

\begin{Parallel}[v]{0.485\textwidth}{0.485\textwidth}
\ParallelLText{
Beátus, quicúmque times Dóminum, * qui ámbulas in viis eius!

Nam labórem mánuum tuárum manducábis, * beátus eris et bene tibi erit.

Uxor tua sicut vitis fructífera * in penetrálibus domus tuæ,

Fílii tui ut súrculi olivárum * circa mensam tuam.

Ecce sic benedícitur viro, * qui timet Dóminum! –

Benedícat tibi Dóminus ex Sion, * ut vídeas prosperitátem Ierúsalem ómnibus diébus vitæ tuæ;

Ut vídeas fílios filiórum tuórum: * pax super Israël!

Glória Patri, et Fílio, * et Spirítui Sancto.

Sicut erat in princípio, et nunc et semper, * et in sǽcula sæculórum. Amen.
}
\ParallelRText{
Błogosławionyś, gdy się boisz Pana, * gdy chodzisz Jego drogami!

Bo z pracy rąk swoich będziesz pożywał, * szczęście osiągniesz i dobrze ci będzie.

Małżonka twoja jak płodny szczep winny, * w zaciszu twojego domu.

Synowie twoi jak oliwne gałązki * dokoła twojego stołu.

Takie błogosławieństwo dla męża, * który boi się Pana! –

Niechaj z Syjonu Pan cię błogosławi, * byś widział pomyślność Jeruzalem przez wszystkie dni twego życia.

Abyś oglądał dzieci twoich synów: * pokój nad Izraelem!

Chwała Ojcu i Synowi, * i Duchowi Świętemu.

Jak była na początku, teraz i zawsze, * i na wieki wieków. Amen.
}
\end{Parallel}

\begin{center}
\textcolor{red}{Po ukończeniu psalmu kapłan śpiewa:}
\end{center}

\begin{Parallel}[v]{0.485\textwidth}{0.485\textwidth}
\ParallelLText{
\versicle Kýrie, eléison.
}
\ParallelRText{
\response Kyrie, elejson.
}
\end{Parallel}

\begin{center}
\textcolor{red}{Wszyscy śpiewają dalej:}
\end{center}

\begin{Parallel}[v]{0.485\textwidth}{0.485\textwidth}
\ParallelLText{
\versicle Christe, eléison. Kýrie, eléison.
}
\ParallelRText{
\response Chryste, elejson. Kyrie, elejson.
}
\end{Parallel}

\begin{center}
\textcolor{red}{Kapłan mówi:}
\end{center}

\begin{Parallel}[v]{0.485\textwidth}{0.485\textwidth}
\ParallelLText{
\versicle Módlmy się, jak nas nauczył Pan nasz Jezus Chrystus.
}
\ParallelRText{
\response Módlmy się, jak nas nauczył Pan nasz Jezus Chrystus.
}
\end{Parallel}

\begin{center}
\textcolor{red}{Wszyscy wspólnie recytują Modlitwę Pańską:}
\end{center}

\begin{Parallel}[v]{0.485\textwidth}{0.485\textwidth}
\ParallelLText{
Pater noster
}
\ParallelRText{
Ojcze nasz.
}
\end{Parallel}

\begin{center}
\textcolor{red}{Potem kapłan śpiewa:}
\end{center}

\begin{Parallel}[v]{0.485\textwidth}{0.485\textwidth}
\ParallelLText{
\versicle Salvos fac servos tuos.

\response Deus meus, sperántes in te.\\

\versicle Mitte eis, Dómine, auxílium de sancto.

\response Et de Sion tuére eos.

\versicle Esto eis, Dómine, turris fortitúdinis.

\response A fácie inimíci.

\versicle Dómine, exáudi oratiónem meam.

\response Et clamor meus ad te véniat.\\

\versicle Dóminus vobíscum.

\response Et cum spíritu tuo.

Orémus. 

Réspice, quǽsumus, Dómine, super hos fámulos tuos: et institútis tuis, quibus propagatiónem humáni géneris ordinásti, benígnus assíste; * ut, qui te auctóre iungúntur, te auxiliánte servéntur. Per Chris-tum, Dóminum nostrum.\\

\response Amen.
}
\ParallelRText{
\versicle Zachowaj sługi swoje.

\response Którzy w Tobie, Boże mój, pokładają nadzieję.

\versicle Ześlij im, Panie, pomoc z przybytku swego.

\response I ze Syjonu racz ich bronić.

\versicle Bądź im, o Panie, wieżą obronną.\\

\response Przeciw zakusom nieprzyjaciela.

\versicle Panie, wysłuchaj modlitwę moją.

\response A wołanie moje niech do Ciebie przyjdzie.

\versicle Pan z wami.

\response I z duchem twoim.

Módlmy się. 

Prosimy Cię, Panie, wejrzyj na te sługi swoje i wspieraj łaskawie związek, który ustanowiłeś dla rozkrzewienia rodzaju ludzkiego. Niech ci, którzy za sprawą Twoją się łączą, * przy Twojej pomocy szczęśliwie wytrwają. Przez Chrystusa, Pana naszego.

\response Amen.
}
\end{Parallel}