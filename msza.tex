%\definecolor{my-color}{gray}{0.35} % szary
%\definecolor{my-color}{rgb}{0.43, 0.21, 0.1} % brązowy
\definecolor{my-color}{rgb}{0.0, 0.26, 0.15} % ciemnozielony

\begin{center}
\Large

\textbf{Msza za nowożeńców}\\[0.3cm] 

\normalsize
\end{center}

\begin{center}
\large
\textbf{Modlitwy u stopni ołtarza}
\normalsize

\textcolor{red}{Kapłan stanąwszy u stopni ołtarza, żegna się znakiem krzyża i wraz z ministrantami odmawia następujące poniżej modlitwy. \textbf{Klękamy wraz z ministrantami i kapłanem.} Wierni śpiewają w tym czasie jakąś pieśń nabożną. Modlitwy u stopni ołtarza to jakby przejście ze świata, z jego zgiełku, do innej rzeczywistości, rzeczywistości sakralnej. Zanim przystąpi się do ołtarza, należy zatem wzbudzić w sobie żal za grzechy i przeprosić za nie.}\\
\textcolor{my-color}{ Teksty zaznaczone tym kolorem odmawia kapłan z ministrantami bądź sam kapłan po cichu. }

\textbf{Psalm 42}
\end{center}

\begin{Parallel}[v]{0.485\textwidth}{0.485\textwidth}
\ParallelLText{
\textcolor{red}{S.}\textcolor{my-color}{ In nómine Patris,\SmallCross et Fílii, et Spíritus Sancti. Amen. Introíbo ad altáre Dei.}

\textcolor{red}{M.} \textcolor{my-color}{ Ad Deum, qui lætíficat iuventútem meam.}

\textcolor{red}{S.} \textcolor{my-color}{Iúdica me, Deus, et discérne causam meam de gente non sancta: ab hómine iníquo et dolóso érue me.}

\textcolor{red}{M.} \textcolor{my-color}{Quia tu es, Deus, fortitudo mea: quare me reppulísti, et quare tristis incédo, dum afflígit me inimícus?}

\textcolor{red}{S.} \textcolor{my-color}{Emítte lucem tuam et veritátem tuam: ipsa me deduxérunt, et adduxérunt in montem sanctum tuum et in tabernácula tua.}

\textcolor{red}{M.} \textcolor{my-color}{Et introíbo ad altáre Dei: ad Deum, qui lætíficat iuventútem meam.}

\textcolor{red}{S.} \textcolor{my-color}{Confitébor tibi in cíthara, Deus, Deus meus: quare tristis es, ánima mea, et quare contúrbas me?}

\textcolor{red}{M.} \textcolor{my-color}{Spera in Deo, quóniam adhuc confitébor illi: salutáre vultus mei, et Deus meus.}

\textcolor{red}{S.} \textcolor{my-color}{Glória Patri, et Fílio, et Spirítui Sancto.}

\textcolor{red}{M.} \textcolor{my-color}{Sicut erat in princípio, et nunc, et semper: et in saecula sæculórum. Amen.}

\textcolor{red}{S.} \textcolor{my-color}{Introíbo ad altáre Dei.}

\textcolor{red}{M.} \textcolor{my-color}{Ad Deum, qui lætíficat iuventútem meam.}
}


\ParallelRText{
\textcolor{red}{S.} \textcolor{my-color}{W imię Ojca\SmallCross i Syna i Ducha Świętego. Amen. Przystąpię do ołtarza Bożego.}

\textcolor{red}{M.} \textcolor{my-color}{Do Boga, który jest weselem i radością moją.}

\textcolor{red}{S.} \textcolor{my-color}{Wymierz mi, Boże, sprawiedliwość i broń sprawy mojej przeciw ludowi niezbożnemu, wybaw mnie od człowieka podstępnego i niegodziwego.}

\textcolor{red}{M.} \textcolor{my-color}{Ty bowiem, Boże, jesteś mocą moją, czemu mię odrzuciłeś? Czemu chodzę smutny, nękany przez wroga?}

\textcolor{red}{S.} \textcolor{my-color}{Ześlij Swą światłość i wierność Swoją: niech one mnie wiodą, niech mnie przywiodą na górę Twą świętą, do Twoich przybytków.}

\textcolor{red}{M.} \textcolor{my-color}{I przystąpię do ołtarza Bożego, do Boga, który jest weselem i radością moją.}

\textcolor{red}{S.} \textcolor{my-color}{I będę Cię chwalił na cytrze, Boże, Boże mój. Duszo ma, czemuś zgnębiona i czemu miotasz się we mnie?}

\textcolor{red}{M.} \textcolor{my-color}{Ufaj Bogu, bo jeszcze wysławiać Go będę, zbawienie mego oblicza i Boga mojego.}

\textcolor{red}{S.} \textcolor{my-color}{Chwała Ojcu i Synowi i Duchowi Świętemu.}

\textcolor{red}{M.} \textcolor{my-color}{Jak była na początku, teraz i zawsze i na wieki wieków. Amen.}

\textcolor{red}{S.} \textcolor{my-color}{Przystąpię do ołtarza Bożego.}

\textcolor{red}{M}. \textcolor{my-color}{Do Boga, który jest weselem i radością moją.}
}
\end{Parallel}

\begin{center}
\textbf{Spowiedź powszechna}
\end{center}

\begin{center}
\textcolor{red}{Wobec Boga i Kościoła powszechnego oskarżamy się publicznie o winy nasze, abyśmy tym głębszą sobie obudzili skruchę. Odmawia ją najpierw celebrans, potem zaś w imieniu wiernych ministranci.\\
Spowiedź kapłana:}
\end{center}

\begin{Parallel}[v]{0.485\textwidth}{0.485\textwidth}
\ParallelLText{

\versicle \textcolor{my-color}{Adiutórium nostrum\SmallCross in nómine Dómini.}

\response \textcolor{my-color}{Qui fecit coelum et terram.}

\textcolor{red}{S.} \textcolor{my-color}{Confíteor Deo omnipoténti, beátæ Maríæ semper Vírgini, beáto Michaéli Archángelo, beáto Ioánni Baptístæ, sanctis Apóstolis Petro et Paulo, ómnibus Sanctis, et vobis, fratres: quia peccávi nimis cogitatióne, verbo et opere: mea culpa, mea culpa, mea máxima culpa. Ideo precor beátam Maríam semper Vírginem, beátum Michaélem Archángelum, beátum Ioánnem Baptístam, sanctos Apóstolos Petrum et Paulum, omnes Sanctos, et vos, fratres, orare pro me ad Dóminum, Deum nostrum.}
}
\ParallelRText{
\versicle \textcolor{my-color}{Wspomożenie nasze\SmallCross w imieniu Pana.}

\response \textcolor{my-color}{Który stworzył niebo i ziemię.}

\textcolor{red}{S.} \textcolor{my-color}{Spowiadam się Bogu wszechmogącemu, Najświętszej Maryi zawsze Dziewicy, świętemu Michałowi Archaniołowi, świętemu Janowi Chrzcicielowi, świętym Apostołom Piotrowi i Pawłowi, wszystkim Świętym i wam, bracia, że bardzo zgrzeszyłem, myślą, mową i uczynkiem: moja wina, moja wina, moja bardzo wielka wina. Przeto błagam Najświętszą Maryję zawsze Dziewicę, świętego Michała Archanioła, świętego Jana Chrzciciela, świętych Apostołów Piotra i Pawła, wszystkich Świętych i was, bracia, o modlitwę za mnie do Pana, Boga naszego.}
}
\end{Parallel}

\begin{center}
\textcolor{red}{Ministranci i wierni proszą za kapłanem.}
\end{center}

\begin{Parallel}[v]{0.485\textwidth}{0.485\textwidth}
\ParallelLText{

\textcolor{red}{M.} \textcolor{my-color}{Misereátur tui omnípotens Deus, et, dimíssis peccátis tuis, perdúcat te ad vitam ætérnam.}

\textcolor{red}{S.} \textcolor{my-color}{Amen.}
}
\ParallelRText{
\textcolor{red}{M.} \textcolor{my-color}{Niech się zmiłuje nad tobą wszechmogący Bóg, a odpuściwszy ci grzechy, doprowadzi cię do życia wiecznego.}

\textcolor{red}{S.} \textcolor{my-color}{Amen.}
}
\end{Parallel}

\begin{center}
\textcolor{red}{Ministranci mówią pochyleni:}
\end{center}

\begin{Parallel}[v]{0.485\textwidth}{0.485\textwidth}
\ParallelLText{
\textcolor{red}{M.} \textcolor{my-color}{Confíteor Deo omnipoténti, beátæ Maríæ semper Vírgini, beáto Michaéli Archángelo, beáto Ioánni Baptístæ, sanctis Apóstolis Petro et Paulo, ómnibus Sanctis, et tibi, pater: quia peccávi nimis cogitatióne, verbo et opere: mea culpa, mea culpa, mea máxima culpa. Ideo precor beátam Maríam semper Vírginem, beátum Michaélem Archángelum, beátum Ioánnem Baptístam, sanctos Apóstolos Petrum et Paulum, omnes Sanctos, et te, pater, orare pro me ad Dóminum, Deum nostrum.}
}
\ParallelRText{
\textcolor{red}{M.} \textcolor{my-color}{Spowiadam się Bogu wszechmogącemu, Najświętszej Maryi zawsze Dziewicy, świętemu Michałowi Archaniołowi, świętemu Janowi Chrzcicielowi, świętym Apostołom Piotrowi i Pawłowi, wszystkim Świętym i tobie, ojcze, że bardzo zgrzeszyłem, myślą, mową i uczynkiem: moja wina, moja wina, moja bardzo wielka wina. Przeto błagam Najświętszą Maryję zawsze Dziewicę, świętego Michała Archanioła, świętego Jana Chrzciciela, świętych Apostołów Piotra i Pawła, wszystkich Świętych i ciebie, ojcze, o modlitwę za mnie do Pana, Boga naszego.}
}
\end{Parallel}

\begin{center}
\textcolor{red}{Kapłan wstawia się za ogółem wiernych:}
\end{center}

\begin{Parallel}[v]{0.485\textwidth}{0.485\textwidth}
\ParallelLText{
\textcolor{red}{S.} \textcolor{my-color}{Misereátur vestri omnípotens Deus, et, dimíssis peccátis vestris, perdúcat vos ad vitam ætérnam.}

\textcolor{red}{M.}\textcolor{my-color}{ Amen.}

\textcolor{red}{S.} \textcolor{my-color}{Indulgéntiam,\SmallCross absolutionem et remissiónem peccatórum nostrórum tríbuat nobis omnípotens et miséricors Dóminus.}

\textcolor{red}{M.} \textcolor{my-color}{Amen.}
}
\ParallelRText{

\textcolor{red}{S.} \textcolor{my-color}{Niech się zmiłuje nad wami wszechmogący Bóg, a odpuściwszy wam grzechy, doprowadzi was do życia wiecznego.}

\textcolor{red}{M.} \textcolor{my-color}{Amen.}

\textcolor{red}{S.} \textcolor{my-color}{Przebaczenia,\SmallCross odpuszczenia i~darowania grzechów niech nam udzieli wszechmogący i miłosierny Pan.}

\textcolor{red}{M.} \textcolor{my-color}{Amen.}
}
\end{Parallel}

\begin{Parallel}[v]{0.485\textwidth}{0.485\textwidth}
\ParallelLText{
\versicle \textcolor{my-color}{Deus, tu convérsus vivificábis nos.}\\

\response \textcolor{my-color}{Et plebs tua lætábitur in te.}

\versicle \textcolor{my-color}{Osténde nobis, Dómine, misericórdiam tuam.}

\response \textcolor{my-color}{Et salutáre tuum da nobis.}

\versicle \textcolor{my-color}{Dómine, exáudi oratiónem meam.}

\response \textcolor{my-color}{Et clamor meus ad te véniat.}\\

\versicle \textcolor{my-color}{Dóminus vobíscum.}

\response \textcolor{my-color}{Et cum spíritu tuo.}

\textcolor{my-color}{Orémus.}
}

\ParallelRText{
\versicle \textcolor{my-color}{Zwróć się ku nam, Boże, i ożyw nas.}

\response \textcolor{my-color}{A lud Twój rozraduje się w Tobie.}

\versicle \textcolor{my-color}{Okaż nam, Panie, miłosierdzie Swoje.}

\response \textcolor{my-color}{I daj nam Swoje zbawienie.}

\versicle \textcolor{my-color}{Panie, wysłuchaj modlitwę moją.}

\response \textcolor{my-color}{A wołanie moje niech do Ciebie przyjdzie.}

\versicle \textcolor{my-color}{Pan z wami.}

\response \textcolor{my-color}{I z duchem twoim.}

\textcolor{my-color}{Módlmy się. }
}
\end{Parallel}

\begin{center}
\textcolor{red}{Kapłan wstępuje po stopniach ołtarza i odmawia modlitwę:}
\end{center}

\begin{Parallel}[v]{0.485\textwidth}{0.485\textwidth}
\ParallelLText{
\textcolor{my-color}{Aufer a nobis, quaesumus, Dómine, iniquitátes nostras: ut ad Sancta sanctórum puris mereámur méntibus introíre. Per Christum, Dóminum nostrum. Amen.}
}

\ParallelRText{
\textcolor{my-color}{Zgładź nieprawości nasze, prosimy Cię, Panie, abyśmy z czystym sercem mogli przystąpić do tajemnic najświętszych. Przez Chrystusa, Pana naszego. Amen.}
}
\end{Parallel}

\begin{center}
\textcolor{red}{Całując ołtarz, w którym zawarte są relikwie świętych:}
\end{center}

\begin{Parallel}[v]{0.485\textwidth}{0.485\textwidth}
\ParallelLText{
\textcolor{my-color}{Orámus te, Dómine, per mérita Sanctórum tuórum, quorum relíquiæ hic sunt, et ómnium Sanctórum: ut indulgére dignéris ómnia peccáta mea. Amen.}
}

\ParallelRText{
\textcolor{my-color}{Prosimy Cię, Panie, przez zasługi Świętych Twoich, których relikwie tutaj się znajdują, oraz wszystkich Świętych, abyś mi raczył odpuścić wszystkie grzechy moje. Amen.}
}
\end{Parallel}

\begin{center}
\textcolor{red}{Podczas Mszy z kadzeniem kapłan błogosławi kadzidło i mówi:}
\end{center}

\begin{Parallel}[v]{0.485\textwidth}{0.485\textwidth}
\ParallelLText{
\textcolor{my-color}{Ab illo benedicáris, in cuius honóre cremabéris. Amen.}
}

\ParallelRText{
\textcolor{my-color}{Niechaj cię Ten błogosławi, na którego cześć spalać się będziesz. Amen.}
}
\end{Parallel}

\begin{center}
\textcolor{red}{Kadzidło, które spala się i dym unoszący się w góre są symbolem naszych modlitw i ofiar. Kapłan okadza ołtarz, następnie ceremoniarz okadza celebransa jako przedstawiciela Chrystusa.}
\end{center}

\begin{center}
\textbf{Introit}
\end{center}

\begin{center}
\textcolor{red}{Kapłan przechodzi na prawą stronę ołtarza i odczytuje Introit.}
\end{center}

\begin{Parallel}[v]{0.485\textwidth}{0.485\textwidth}
\ParallelLText{
\textcolor{my-color}{Deus\SmallCross Israel coniugat vos: et ipse sit vobiscum, qui misertus est duo bus unicis: et nunc, Domine, fac eos plenius benedicere te.}

\textcolor{my-color}{Beati omnes, qui timent Dominum: qui ambulant in viis eius. Gloria Patri et Filio et Spiritu Sa-ncto. Sicut erat in principio, et nunc et simper et in saecula saeculorum. Amen.}
}

\ParallelRText{
\textcolor{my-color}{Bóg Izraela niech was połączy, a który nad dwojgiem oblubieńców się zmiłował nich będzie także z wami. A teraz dozwól Panie, aby jeszcze gorliwiej Cię wielbili.}

\textcolor{my-color}{Błogosławieni wszyscy, którzy się boją Pana, którzy chodzą Jego drogami. Amen.}
}
\end{Parallel}

\begin{center}
\textbf{Kyrie}

\textcolor{red}{Kyrie śpiewają wierni wraz z organistą }

\end{center}

\begin{Parallel}[v]{0.485\textwidth}{0.485\textwidth}
\ParallelLText{
\textcolor{red}{S.} Kýrie, eléison.

\textcolor{red}{M.} Kýrie, eléison.

\textcolor{red}{S.} Kýrie, eléison.

\textcolor{red}{M.} Christe, eléison.

\textcolor{red}{S.} Christe, eléison.

\textcolor{red}{M.} Christe, eléison.

\textcolor{red}{S.} Kýrie, eléison.

\textcolor{red}{M.} Kýrie, eléison.

\textcolor{red}{S.} Kýrie, eléison.
}

\ParallelRText{
\textcolor{red}{S.} Panie, zmiłuj się.

\textcolor{red}{M.} Panie, zmiłuj się.

\textcolor{red}{S.} Panie, zmiłuj się.

\textcolor{red}{M.} Chryste, zmiłuj się.

\textcolor{red}{S.} Chryste, zmiłuj się.

\textcolor{red}{M.} Chryste, zmiłuj się.

\textcolor{red}{S.} Panie, zmiłuj się.

\textcolor{red}{M.} Panie, zmiłuj się.

\textcolor{red}{S.} Panie, zmiłuj się.
}
\end{Parallel}

\begin{center}
\textcolor{red}{\textbf{Wierni wstają.}}
\end{center}

\begin{center}
\textbf{Gloria}
\end{center}

\begin{Parallel}[v]{0.485\textwidth}{0.485\textwidth}
\ParallelLText{
Gloria in excelsis Deo Et in terra pax homínibus bonæ voluntátis. Laudámus te. Benedícimus te. Adorámus te. Glorificámus te. Grátias ágimus tibi propter magnam glóriam tuam. Dómine Deus, Rex coeléstis, Deus Pater omnípotens. Dómine Fili unigénite, Iesu Christe. Dómine Deus, Agnus Dei, Fílius Patris. Qui tollis peccáta mundi, miserére nobis. Qui tollis peccáta mundi, súscipe deprecatiónem nostram. Qui sedes ad déxteram Patris, miserére nobis. Quóniam tu solus Sanctus. Tu solus Dóminus. Tu solus Altíssimus, Iesu Christe. Cum Sancto Spíritu\SmallCross in glória Dei Patris. Amen.
}

\ParallelRText{
Chwała na wysokości Bogu, a na ziemi pokój ludziom dobrej woli. Chwalimy Cię. Błogosławimy Cię. Wielbimy Cię.Wysławiamy Cię. Dzięki Ci składamy, bo wielka jest chwała Twoja. Panie Boże, Królu nieba, Boże Ojcze wszechmogący. Panie, Synu Jednorodzony, Jezu Chryste. Panie Boże, Baranku Boży, Synu Ojca. Który gładzisz grzechy świata, zmiłuj się nad nami. Który gładzisz grzechy świata, przyjm błaganie nasze. Który siedzisz po prawicy Ojca, zmiłuj się nad nami. Albowiem tylko Tyś jest święty. Tylko Tyś jest Panem. Tylko Tyś Najwyższy, Jezu Chryste. Z Duchem Świętym\SmallCross w chwale Boga Ojca. Amen.
}
\end{Parallel}

\begin{center}
\textbf{Kolekta}
\end{center}

\begin{Parallel}[v]{0.485\textwidth}{0.485\textwidth}
\ParallelLText{
Exaudi nos, omnipotens et misericors Deus: ut quod nostro ministratur officio, tua benediction potius impleatur. Per Dominum nostrum Iesum Christum.
}

\ParallelRText{
Wysłuchaj nas wszechmogący i miłosierny Boże, racz jak najobficiej pobłogosławić ten związek małżeński przy posługiwaniu naszym zawarty. Przez Pana naszego.
}
\end{Parallel}

\begin{center}
\textbf{Lekcja}
\end{center}

\begin{center}
\textcolor{red}{\textbf{Wierni na lekcję mogą usiąść.}}
\end{center}

\begin{Parallel}[v]{0.485\textwidth}{0.485\textwidth}
\ParallelLText{
S. Lectio beati Pauli Apostoli ad Ephesios: (ad Ephesois, 5, 22-33) Mulieres viris suis subditæ sint, sicut Domino:  quoniam vir caput est mulieris, sicut Christus caput est Ecclesiæ: ipse, salvator corporis ejus. Sed sicut Ecclesia subjecta est Christo, ita et mulieres viris suis in omnibus. Viri, diligite uxores vestras, sicut et Christus dilexit Ecclesiam, et seipsum tradidit pro ea, ut illam sanctificaret, mundans lavacro aquæ in verbo vitæ, ut exhiberet ipse sibi gloriosam Ecclesiam, non habentem maculam, aut rugam, aut aliquid huiusmodi, sed ut sit sancta et immaculata. Ita et viri debent diligere uxores suas ut corpora sua. Qui suam uxorem diligit, seipsum diligit. Nemo enim umquam carnem suam odio habuit: sed nutrit et fovet eam, sicut et Christus Ecclesiam: quia membra sumus corporis ejus, de carne ejus et de ossibus ejus. Propter hoc relinquet homo patrem et matrem suam, et adhærebit uxori suæ, et erunt duo in carne una. Sacramentum hoc magnum est, ego autem dico in Christo et in Ecclesia. Verumtamen et vos singuli, unusquisque uxorem suam sicut seipsum diligat: uxor autem timeat virum suum.
}

\ParallelRText{
Lekcja z listu św. Pawła Apostoła do Efezjan (Efez. 5, 22-33): Żony niech będą poddane swoim mężom jak Panu, ponieważ mąż jest głową żony, jak Chrystus jest głową Kościoła: on, Zbawiciel ciała jego. Lecz jak Kościół poddany jest Chrystusowi, tak i żony swoim mężom we wszystkim. Mężowie, miłujcie żony wasze, jak i Chrystus umiłował Kościół i wydał samego siebie za niego, aby go uświęcić, oczyściwszy go obmyciem wody w słowie życia, aby sam sobie przysposobił Kościół chwalebny, nie mający zmazy ani zmarszczki, albo czegoś podobnego, ale żeby był święty i niepokalany. Tak i mężowie powinni miłować żony swoje jak swoje ciało. Kto miłuje żonę swoją, miłuje samego siebie. Nikt bowiem nigdy nie miał w nienawiści ciała swego, ale żywi i pielęgnuje je, jak i Chrystus Kościół; gdyż jesteśmy członkami ciała jego, z ciała jego i z kości jego. ,,Dlatego opuści człowiek ojca i matkę swoją i złączy się z żoną swoją i będą dwoje w jednym ciele.'' Jest to wielka tajemnica, a ja mówię w Chrystusie i w Kościele. Wszakże i każdy z was z osobna niech miłuje żonę swoją jak samego siebie, a żona niech się boi męża swego.
}
\end{Parallel}

\begin{center}
\textcolor{red}{Po skończonym czytaniu odpowiada się:}
\end{center}

\begin{Parallel}[v]{0.485\textwidth}{0.485\textwidth}
\ParallelLText{
\response Deo gratias.
}

\ParallelRText{
\response Bogu dzięki.
}
\end{Parallel}

\begin{center}
\textbf{Gradułał}
\end{center}

\begin{center}
\textcolor{red}{Po Lekcji chór śpiewa (względnie kapłan czyta) tekst śpiewu międzylekcyjnego zwanego Graduałem. Tekst ów jest wyjęty z Psałterza, streszcza pobożne uczucia, jakie nam czytanie Pisma Świętego nasunęło. Alleluja - słowo hebrajskie, oznaczające ,,śpiewajcie Panu'' - jest okrzykiem radości płynącej z serca Kościoła Świętego na skutek dobrodziejstw Bożych. Wierni w tym czasie siedzą. Nie wstają na ,,Alleluja'', wbrew nowym obyczajom.}
\end{center}

\begin{Parallel}[v]{0.485\textwidth}{0.485\textwidth}
\ParallelLText{
\textcolor{my-color}{Uxor tua sicut vitis abundans in lateribus domus tuae.}

\versicle \textcolor{my-color}{Filii tui sicut novellae olivarum in circuitu mensae tuae.}

\textcolor{my-color}{Alleluia, alleluia. Mittat vobis Dominus auxulium de sancto: et de Sion tueatur vos. Alleluia.}
}

\ParallelRText{
\textcolor{my-color}{Żona twoja jako winorośl płodna, rosnąca na ścianie twego domu.}

\versicle \textcolor{my-color}{Dzieci twe jak gałązki oliwne wokoło twego stołu.}

\textcolor{my-color}{Alleluja, alleluja. Niech Pan ześle wam pomoc z świątyni i ze Syjonu niechaj was broni. Alleluja.}
}
\end{Parallel}


\begin{center}
\textbf{Ewangelia}
\end{center}

\begin{center}
\textcolor{red}{Po odczytaniu gradułału kapłan modli się przed oczytaniem Ewangelii:}
\end{center}

\begin{Parallel}[v]{0.485\textwidth}{0.485\textwidth}
\ParallelLText{
Munda cor meum, ac labia mea, omnípotens Deus, qui labia Isaíæ Prophétæ cálculo mundásti igníto: ita me tua grata miseratióne dignáre mundáre, ut sanctum Evangélium tuum digne váleam nuntiáre. Per Christum, Dóminum nostrum. Amen.

Iube, Dómine, benedícere. Dóminus sit in corde meo et in lábiis meis: ut digne et competénter annúntiem Evangélium suum. Amen.
}

\ParallelRText{
Oczyść serce i wargi moje, wszechmogący Boże, któryś wargi proroka Izajasza oczyścił kamykiem ognistym. W łaskawym zmiłowaniu Swoim racz mię tak oczyścić, abym godnie zdołał głosić Twą świętą Ewangelię. Przez Chrystusa, Pana naszego. Amen.

Racz pobłogosławić, Panie. Pan niech będzie w sercu moim i na wargach moich, bym godnie i należycie głosił Jego Ewangelię. Amen.
}
\end{Parallel}

\begin{center}
\textcolor{red}{W tym czasie zaś ministrant przenosi mszał na drugą stronę ołtarza. Jak niemal każdy gest we Mszy św. przeniesienie mszału ma głęboką symbolikę - symbolizuje przeniesienie głoszenia Ewangelii Chrystusowej na pogan; albowiem najpierw była ona głoszona żydom, ci jednak ją odrzucili zanurzając się w błędy faryzeizmu i gardząc Bożym przymerzem, poganie zaś z radością przyjęli zbawczą, Boską naukę. W Ewangelii przemawia do nas sam Pan Jezus. \textbf{Wierni wstają.}}
\end{center}

\begin{Parallel}[v]{0.485\textwidth}{0.485\textwidth}
\ParallelLText{
\versicle Dóminus vobíscum.

\response Et cum spíritu tuo.

Sequéntia\SmallCross sancti Evangélii secúndum Lucam

\response Gloria tibi, Domine!
}

\ParallelRText{
\versicle Pan z wami.

\response I z duchem twoim.

Ciąg dalszy\SmallCross Ewangelii świętej według Łukasza.

\response Chwała Tobie Panie.
}
\end{Parallel}

\begin{Parallel}[v]{0.485\textwidth}{0.485\textwidth}
\ParallelLText{
In illo tempore accesserunt ad eum pharisæi tentantes eum, et dicentes: Si licet homini dimittere uxorem suam, quacumque ex causa? Qui respondens, ait eis : Non legistis, quia qui fecit hominem ab initio, masculum et feminam fecit eos? Et dixit: Propter hoc dimittet homo patrem, et matrem, et adhærebit uxori suæ, et erunt duo in carne una.
}

\ParallelRText{
(Mat. 19, 3-6) Onego czasu przyszli do Jezusa faryzeusze, kusząc go i mówiąc: Czy godzi się człowiekowi opuścić żonę swoją z jakiejkolwiek przyczyny? A on odpowiadając rzekł im: Nie czytaliście, że ten który stworzył człowieka na początku, mężczyzną i niewiastą stworzył ich? I rzekł: Dlatego opuści człowiek ojca i matkę, i złączy się z żoną swoją, i będą dwoje w jednym ciele. A tak już nie są dwoje, ale jedno ciało. Co więc Bóg złączył, człowiek niech nie rozłącza.
}
\end{Parallel}

\begin{center}
\textcolor{red}{Po skończonej Ewangelii odpowiada się:}
\end{center}

\begin{Parallel}[v]{0.485\textwidth}{0.485\textwidth}
\ParallelLText{
\response Laus tibi, Christe.
}

\ParallelRText{
\response Chwała Tobie, Chryste.
}
\end{Parallel}

\begin{center}
\textcolor{red}{Kapłan po cichu mówi:}
\end{center}

\begin{Parallel}[v]{0.485\textwidth}{0.485\textwidth}
\ParallelLText{
\textcolor{red}{S.} Per Evangelica dicta, deleantur nostra delicta.
}

\ParallelRText{
\textcolor{red}{S.} Niech słowa Ewangelii zgładzą nasze grzechy.
}
\end{Parallel}

\begin{center}
\large
\textbf{Ofiarowanie}
\normalsize
\end{center}

\begin{center}
\textcolor{red}{Kapłan całuje ołtarz i pozdrawia wiernych wzywając wszystkich do modlitwy. Modlitwy ofiarowania są odmawiane przez księdza cicho. Są to przepiękne modlitwy, w których kapłan ofiaruje Bogu wino i chleb - te same, które za chwilę staną się Ciałem i Krwią Pana Naszego Jezusa Chrystusa. To ofiarowanie jest jakby przygotowaniem właściwej ofiary Mszy świętej, która dokona się nieco później.}
\end{center}

\begin{Parallel}[v]{0.485\textwidth}{0.485\textwidth}
\ParallelLText{
\versicle Dóminus vobíscum.

\response Et cum spíritu tuo. 

Orémus
}
\ParallelRText{
\versicle  Pan z wami.

\response I z duchem twoim.

Módlmy się.
}
\end{Parallel}

\begin{center}
\textcolor{red}{\textbf{Wierni siadają.} W czasie ofiarowania wierni śpiewają pieśń. W chwili ofiarowania oddajmy Panu Jezusowi całe nasze życie, wszystkie poczynania, troski i radości; niech On złoży te dary nasze przed Tronem Bożym w niebie.}
\end{center}

\begin{center}
\textbf{Antyfona na Ofiarowanie}
\end{center}

\begin{Parallel}[v]{0.485\textwidth}{0.485\textwidth}
\ParallelLText{
\textcolor{my-color}{In te speravi, Domine: dixi: Tu es Deus meus: in manibus tuis tempora mea.}
}
\ParallelRText{
\textcolor{my-color}{W Tobie, o Panie, ufność pokładam. Tyś moim Bogiem, mówię, los mój w Twoim jest ręku.}
}
\end{Parallel}

\begin{center}
\textbf{Ofiarowanie Chleba}
\end{center}

%\usepackage[x11names]{xcolor}

%\definecolor{my-color}{gray}{0.35} % szary
%\definecolor{my-color}{rgb}{0.43, 0.21, 0.1} % brązowy

\begin{center}
\textcolor{red}{Kapłan odkrywa kielich, ministrant jeden raz dzwoni. Następnie wznosząc patenę kapłan ofiarowuje chleb.}
\textcolor{my-color}{(Odmawiane przez kapłana po cichu)}
\end{center}

\begin{Parallel}[v]{0.485\textwidth}{0.485\textwidth}
\ParallelLText{
\textcolor{my-color}{Suscipe, sancte Pater, omnipotens ætérne Deus, hanc immaculátam hóstiam, quam ego indígnus fámulus tuus óffero tibi Deo meo vivo et vero, pro innumerabílibus peccátis, et offensiónibus, et neglegéntiis meis, et pro ómnibus circumstántibus, sed et pro ómnibus fidélibus christiánis vivis atque defúnctis: ut mihi, et illis profíciat ad salútem in vitam ætérnam. Amen.}
}
\ParallelRText{
\textcolor{my-color}{Ojcze święty, wszechmogący, wieczny Boże, przyjmij tę nieskalaną hostię, którą ja, niegodny sługa Twój, ofiaruję Tobie, Bogu mojemu żywemu i prawdziwemu, za niezliczone grzechy, przewinienia i zaniedbania swoje i za wszystkich tu obecnych, a także za wszystkich wiernych chrześcijan żywych i umarłych, aby mnie oraz im przyczyniła się do zbawienia wiecznego. Amen.}
}
\end{Parallel}

\begin{center}
\textcolor{red}{Czyniąc pateną znak krzyża, składa chleb na ołtarzu.}
\end{center}

\begin{center}
\textbf{Przygotowanie wina i wody}
\end{center}

\begin{center}
\textcolor{red}{Ministranci przynoszą ampułki z winem i wodą. Kapłan przeszedłszy na stronę Lekcji wlewa do kielicha wino i kilka kropel wody, którą błogo-sławi. To połączenie wina i wody czyni na pamiątkę krwi i wody, którą spłynęła z boku Chrystusa. Woda jest tez symbolem wiernych, którzy wraz z Chrystusem, głową Kościoła, są złączeni jak woda z winem i z Nim ofiarowują się Bogu Ojcu..}
\end{center}

\begin{Parallel}[v]{0.485\textwidth}{0.485\textwidth}
\ParallelLText{
\textcolor{my-color}{Deus, qui humánæ substántiæ dignitátem mirabíliter condidísti, et mirabílius reformásti: da nobis per huius aquæ et vini mystérium, eius divinitátis esse consórtes, qui humanitátis nostræ fíeri dignátus est párticeps, Iesus Christus, Fílius tuus, Dóminus noster: Qui tecum vivit et regnat in unitáte Spíritus Sancti Deus: per ómnia saecula sæculórum. Amen.}
}
\ParallelRText{
\textcolor{my-color}{Boże, który godność natury ludzkiej przedziwnie stworzyłeś, a jeszcze przedziwnej naprawiłeś: daj nam przez tajemnicę tej wody i wina uczestniczyć w Bóstwie Tego, który raczył stać się uczestnikiem naszego człowieczeństwa, Jezus Chrystus, Twój Syn, a nasz Pan. Który z Tobą żyje i króluje w jedności Ducha Świętego Bóg, przez wszystkie wieki wieków. Amen.}
}
\end{Parallel}

\begin{center}
\textbf{Ofiarowanie wina}
\end{center}

\begin{center}
\textcolor{red}{Na środku ołtarza, kapłan podnosi kielich, a następnie czyni nim znak krzyża nad ołtarzem.}
\end{center}

\begin{Parallel}[v]{0.485\textwidth}{0.485\textwidth}
\ParallelLText{
\textcolor{my-color}{Offérimus tibi, Dómine, cálicem salutáris, tuam deprecántes cleméntiam: ut in conspéctu divínæ maiestátis tuæ, pro nostra et totíus mundi salute, cum odóre suavitátis ascéndat. Amen.}
}
\ParallelRText{
\textcolor{my-color}{Ofiarujemy Ci, Panie, kielich zbawienia, i błagamy łaskawość Twoją, aby jako woń miła wzniósł się przed oblicze Boskiego majestatu Twego za zbawienie nasze i całego świata. Amen.}
}
\end{Parallel}

\begin{center}
\textbf{Polecenie ofiar}
\end{center}

\begin{center}
\textcolor{red}{Kapłan pochylony odmawia modlitwę Azariasza w piecu gorejącym (Dan. III, 39-40).}
\end{center}

\begin{Parallel}[v]{0.485\textwidth}{0.485\textwidth}
\ParallelLText{
\textcolor{my-color}{In spíritu humilitátis et in ánimo contríto suscipiámur a te, Dómine: et sic fiat sacrifícium nostrum in conspéctu tuo hódie, ut pláceat tibi, Dómine Deus.}

\textcolor{my-color}{Veni, sanctificátor omnípotens ætérne Deus: et bene\SmallCross dic hoc sacrifícium, tuo sancto nómini præparátum.}
}
\ParallelRText{
\textcolor{my-color}{Przyjmij nas, Panie, którzy stajemy przed Tobą w duchu pokory i z sercem skruszonym, a ofiara nasza tak niech się dzisiaj dokona przed obliczem Twoim, aby się podobała Tobie, Panie Boże.}

\textcolor{my-color}{Przyjdź, Uświęcicielu, wszechmogący, wieczny Boże, i pobłogosław tę ofiarę, przygotowaną Twemu świętemu Imieniu.}
}
\end{Parallel}

\begin{center}
\textcolor{red}{We Mszy z kadzidłem w tym miejscu następuje okadzenie. Po zasypaniu ksiądz okadza ołtarz, potem zaś turyfer okadza celebransa. Następnie odchodzi na środek i okadza ministrantów (jeśli wśród wiernych siedzą duchowni to podchodzi do każdego i również go okadza), następnie odwraca się do ludu wiernego i okadza ten lud. Na okadzenie się wstaje, gdy ministrant odwróci się z powrotem do ołtarza, można ponownie usiąść.}
\end{center}

\begin{center}
\textbf{Lavabo - Umycie rąk}
\end{center}

\begin{center}
\textcolor{red}{Ministranci przynoszą wodę, ręczniczek i miseczkę do obmycia palców kapłana. Kapłan po okadzeniu lub zaraz po poleceniu ofiar przechodzi na stronę Lekcji i umywa palce rąk. Jest to przypomnienie, że należy obudzić skruchę, aby móc przystąpić do ofiary z duszą czystą. Obmywając palce celebrans modli się słowami Psalmu 25}
\end{center}

\begin{Parallel}[v]{0.485\textwidth}{0.485\textwidth}
\ParallelLText{
\textcolor{my-color}{Lavábo inter innocéntes manus meas: et circúmdabo altáre tuum. Dómine: Ut áudiam vocem laudis, et enárrem univérsa mirabília tua. Dómine, diléxi decórem domus tuæ et locum habitatiónis glóriæ tuæ. Ne perdas cum ímpiis, Deus, ánimam meam, et cum viris sánguinum vitam meam: In quorum mánibus iniquitátes sunt: déxtera eórum repléta est munéribus. Ego autem in innocéntia mea ingréssus sum: rédime me et miserére mei. Pes meus stetit in dirécto: in ecclésiis benedícam te, Dómine. Glória Patri, et Fílio, et Spirítui Sancto. Sicut erat in princípio, et nunc, et semper, et in saecula saeculórum. Amen.}
}
\ParallelRText{
\textcolor{my-color}{Umywam ręce moje na znak niewinności i obchodzę ołtarz Twój, Panie. By jawnie ogłaszać chwałę i rozpowiadać wszystkie cuda Twoje. Miłuję, Panie, siedzibę Twego domu i miejsce przybytku Twej chwały. Nie zabieraj z grzesznymi mej duszy i życia mego z mężami krwawymi. W ręku ich zbrodnia, a ich prawica pełna jest przekupstwa. Ja zaś postępuję w niewinności mojej. Wyzwól mię, zmiłuj się nade mną. Na drodze równej stoi stopa moja. Na zgromadzeniach będę błogosławił Panu. Chwała Ojcu, i Synowi i Duchowi Świętemu. Jak była na początku, teraz i zawsze i na wieki wieków. Amen.}
}
\end{Parallel}

\begin{center}
\textbf{Polecenie ofiar Trójcy Świętej}
\end{center}

\begin{Parallel}[v]{0.485\textwidth}{0.485\textwidth}
\ParallelLText{
\textcolor{my-color}{Súscipe, sancta Trinitas, hanc oblatiónem, quam tibi offérimus ob memóriam passiónis, resurrectiónis, et ascensiónis Iesu Christi, Dómini nostri: et in honórem beátæ Maríæ semper Vírginis, et beáti Ioannis Baptistæ, et sanctórum Apostolórum Petri et Pauli, et istórum et ómnium Sanctórum: ut illis profíciat ad honórem, nobis autem ad salútem: et illi pro nobis intercédere dignéntur in coelis, quorum memóriam ágimus in terris. Per eúndem Christum, Dóminum nostrum. Amen.}
}
\ParallelRText{
\textcolor{my-color}{Przyjmij, Trójco Święta, tę ofiarę, którą Ci składamy na pamiątkę Męki, Zmartwychwstania i Wniebowstąpienia Jezusa Chrystusa, Pana naszego, oraz na cześć Najświętszej Maryi zawsze Dziewicy, świętego Jana Chrzciciela, świętych Apostołów Piotra i Pawła i tych [których relikwie tutaj się znajdują], i wszystkich Świętych: aby im przyniosła cześć, a nam zbawienie, i aby w niebie raczyli orędować za nami ci, których pamiatkę obchodzimy na ziemi. Przez tegoż Chrystusa Pana Naszego. Amen.}
}
\end{Parallel}

\begin{center}
\textbf{Wezwanie do modlitwy i Sekreta}
\end{center}

\begin{center}
\textcolor{red}{Kapłan zwraca się do wiernych i wzywa ich do modlitwy. Ministrant odpowiada:}
\end{center}

\begin{Parallel}[v]{0.485\textwidth}{0.485\textwidth}
\ParallelLText{
\textcolor{red}{S.} \textcolor{my-color}{Oráte, fratres: ut meum ac vestrum sacrifícium acceptábile fiat apud Deum Patrem omnipoténtem.}
}

\ParallelRText{
\textcolor{red}{S.} \textcolor{my-color}{Módlcie się, bracia, aby moją i waszą ofiarę przyjął Bóg Ojciec wszechmogący.}
}
\end{Parallel}

\begin{center}
\textcolor{red}{Ministrant odpowiada:}
\end{center}

\begin{Parallel}[v]{0.485\textwidth}{0.485\textwidth}
\ParallelLText{
\textcolor{red}{M.} \textcolor{my-color}{Suscípiat Dóminus sacrifícium de mánibus tuis ad laudem et glóriam nominis sui, ad utilitátem quoque nostram, totiúsque Ecclésiæ suæ sanctæ.}

\textcolor{red}{S.} \textcolor{my-color}{Amen.}
}

\ParallelRText{
\textcolor{red}{M.} \textcolor{my-color}{Niech Pan przyjmie ofiarę z rąk twoich na cześć i chwałę imienia Swojego, ku pożytkowi również naszemu i całego swego Kościoła świętego.}

\response \textcolor{my-color}{Amen.}
}
\end{Parallel}

\begin{center}
\textcolor{red}{Po czym kapłan, z rękami nad ofiarą, odmawia modlitwę zwaną sekretą:}
\end{center}

\begin{Parallel}[v]{0.485\textwidth}{0.485\textwidth}
\ParallelLText{
\textcolor{my-color}{Suscipe, qaesumus, Domine, pro sacra connubii lege minus oblatum: et cuius largitor es operis, esto dispositor. Per Dominum nostrum Jesum Christum filiium tuum, qui tecum vivit et regnat in unitate Spiritus Sancti Deus.}
}

\ParallelRText{
\textcolor{my-color}{Przyjmij, prosimy, Panie ofiarę którą Ci składamy w intencji tego świętego związku małżeńskiego i kieruj dziełem, któreś ustanowił. Przez Pana naszego Jeżusa Chrystusa, który z Tobą żyje i króluje w jedności Ducha Świętego.}
}
\end{Parallel}

\begin{center}
\textcolor{red}{Po czym kapłan głośno śpiewa:}
\end{center}


\begin{Parallel}[v]{0.485\textwidth}{0.485\textwidth}
\ParallelLText{
Per omnia saecula saeculorum. 

\textcolor{red}{S.} Amen.
}

\ParallelRText{
Bóg przez wszystkie wieki wieków.

\response Amen.
}
\end{Parallel}


\begin{center}
\textbf{Prefacja}
\end{center}

\begin{center}
\textcolor{red}{\textbf{Wierni wstają.}}
\end{center}

\begin{Parallel}[v]{0.485\textwidth}{0.485\textwidth}
\ParallelLText{
\versicle Dóminus vobíscum.

\response Et cum spíritu tuo.

\versicle Sursum corda.

\response Habémus ad Dóminum.

\versicle Grátias agámus Dómino, Deo nostro.

\response Dignum et iustum est.
}

\ParallelRText{
\versicle Pan z wami.

\response I z duchem twoim.

\versicle W górę serca.

\response Wznieśliśmy je ku Panu.

\versicle Dzięki składajmy Panu, Bogu naszemu.

\response Godne to i sprawiedliwe.
}
\end{Parallel}

\begin{center}
\textbf{Święty}
\end{center}

\begin{center}
\textcolor{red}{\textbf{Ministrant dzwoni, wierni klękają.}}
\end{center}

\begin{Parallel}[v]{0.485\textwidth}{0.485\textwidth}
\ParallelLText{
Sanctus, Sanctus, Sanctus Dóminus, Deus Sábaoth. Pleni sunt coeli et terra glória tua. Hosánna in excélsis. Benedíctus\SmallCross, qui venit in nómine Dómini. Hosánna in excélsis.
}

\ParallelRText{
Święty, Święty, Święty Pan Bóg Zastępów! Pełne są niebiosa i ziemia chwały Twojej. Hosanna na wysokości. Błogosławiony\SmallCross, który idzie w imię Pańskie. Hosanna na wysokości.
}
\end{Parallel}

\begin{center}
\large
\textbf{Kanon}
\normalsize
\end{center}

\begin{center}
\textcolor{red}{Modlitwy Kanonu, którego pochodzenie sięga pierwszych wieków Kościoła, odzwierciedlają dokładnie myśl Zbawiciela i Apostołów. Przez wieki modlitwy te nie zmieniały się i trwały w Mszy rzymskiej, dopiero papież Jan XXIII dokonał lekkiej modyfikacji poprzez dodanie do Kanonu imienia św. Józefa. Teraz dokonuje się właściwa ofiara, Chrystus Pan realnie obecny w postaciach Eucharystycznych ofiaruje się tak jak niegdyś w Jerozolimie. Tym razem jednak w sposób bezkrwawy. Podczas Kanonu dobrze jest rozważać Mękę i Ofiarę Chrystusową. Modlitwy Kanonu Mszy są odmawiane przez kapłana po cichu.}
\end{center}

\begin{center}
\textbf{Modlitwa wstawiennicza}
\end{center}

\begin{Parallel}[v]{0.485\textwidth}{0.485\textwidth}
\ParallelLText{
\textcolor{my-color}{Te igitur, clementíssime Pater, per Iesum Christum, Fílium tuum, Dóminum nostrum, súpplices rogámus, ac pétimus, uti accepta habeas et benedícas, hæc \SmallCross dona, hæc \SmallCross múnera, hæc \SmallCross sancta sacrifícia illibáta.}
}

\ParallelRText{
\textcolor{my-color}{Ciebie przeto, najmiłościwszy Ojcze, pokornie i usilnie błagamy przez Jezusa Chrystusa Syna Twego, Pana naszego, abyś łaskawie przyjął i błogosławił te dary, te daniny, te święte ofiary nieskalane.}
}
\end{Parallel}

\begin{center}
\textcolor{red}{Za Kościół wojujący:}
\end{center}

\begin{Parallel}[v]{0.485\textwidth}{0.485\textwidth}
\ParallelLText{
\textcolor{my-color}{In primis, quæ tibi offérimus pro Ecclésia tua sancta cathólica: quam pacificáre, custodíre, adunáre et régere dignéris toto orbe terrárum: una cum fámulo tuo Papa nostro et Antístite nostro et ómnibus orthodóxis, atque cathólicæ et apostólicae fídei cultóribus. }
}

\ParallelRText{
\textcolor{my-color}{Składamy Ci je przede wszystkim za Kościół Twój święty katolicki, racz go darzyć pokojem, strzec, jednoczyć i rządzić nim na całym okręgu ziemskim wraz ze sługą Twoim, papieżem naszym N. i biskupem naszym N., jak również ze wszystkimi wiernymi stróżami wiary katolickiej i apostolskiej.}
}
\end{Parallel}

\begin{center}
\textcolor{red}{Za uczestników:}
\end{center}

\begin{Parallel}[v]{0.485\textwidth}{0.485\textwidth}
\ParallelLText{
\textcolor{my-color}{Meménto, Dómine, famulórum famularúmque tuarum N. et N. et ómnium circumstántium, quorum tibi fides cógnita est et nota devótio, pro quibus tibi offérimus: vel qui tibi ófferunt hoc sacrifícium laudis, pro se suísque ómnibus: pro redemptióne animárum suárum, pro spe salútis et incolumitátis suæ: tibíque reddunt vota sua ætérno Deo, vivo et vero.}
}

\ParallelRText{
\textcolor{my-color}{Pomnij, Panie, na sługi i służebnice Twoje N. i N. i na wszystkich tu obecnych, których wiara jest Ci znana i oddanie jawne. Za nich to składamy Ci tę ofiare chwały i oni sami Tobie ją zanoszą za siebie oraz wszystkich swoich, w intencji odkupienia dusz swoich, w nadziei zbawienia i pomyślności, modły też swoje ślą do Ciebie, Boga wiecznego, żywego i prawdziwego.}
}
\end{Parallel}

\begin{center}
\textcolor{red}{W łączności z Kościołem tryumfującym:}
\end{center}

\begin{Parallel}[v]{0.485\textwidth}{0.485\textwidth}
\ParallelLText{
\textcolor{my-color}{Communicántes, et memóriam venerántes, in primis gloriósæ semper Vírginis Maríæ, Genetrícis Dei et Dómini nostri Iesu Christi: sed et beati Ioseph, eiusdem Virginis Sponsi,
et beatórum Apostolórum ac Mártyrum tuórum, Petri et Pauli, Andréæ, Iacóbi, Ioánnis, Thomæ, Iacóbi, Philíppi, Bartholomaei, Matthaei, Simónis et Thaddaei: Lini, Cleti, Cleméntis, Xysti, Cornélii, Cypriáni, Lauréntii, Chrysógoni, Ioánnis et Pauli, Cosmæ et Damiáni: et ómnium Sanctórum tuórum; quorum méritis precibúsque concédas, ut in ómnibus protectiónis tuæ muniámur auxílio. Per eúndem Christum, Dóminum nostrum. Amen.}
}

\ParallelRText{
\textcolor{my-color}{Zjednoczeni w Świętych Obcowaniu, ze czcią wspominamy najpierw chwalebną zawsze Dziewicę Maryję, Matkę Boga i Pana naszego Jezusa Chrystusa: a także Świętego Józefa, Oblubieńca Najświętszej Dziewicy,
oraz świętych Apostołów i Męczenników Twoich: Piotra i Pawła, Andrzeja, Jakuba, Jana, Tomasza, Jakuba, Filipa, Bartłomieja, Mateusza, Szymona, i Tadeusza, Linusa, Kleta, Klemensa, Sykstusa, Korneliusza, Cypriana, Wawrzyńca, Chryzogona, Jana i Pawła, Kosmę i Damiana, i wszystkich Świętych Twoich. Dla ich zasług i modlitw racz nas we wszystkim otaczać Swą przemożną opieką. Przez tegoż Chrystusa, Pana naszego. Amen.}
}
\end{Parallel}

\begin{center}
\textbf{Prośba o przyjęcie ofiary}
\end{center}

\begin{center}
\textcolor{red}{Kapłan wyciąga dłonie nad Hostią i kielichem, ministrant dzwoni.}
\end{center}

\begin{Parallel}[v]{0.485\textwidth}{0.485\textwidth}
\ParallelLText{
\textcolor{my-color}{Hanc igitur oblatiónem servitutis nostræ, sed et cunctae famíliæ tuæ,
quaesumus, Dómine, ut placátus accípias: diésque nostros in tua pace dispónas, atque ab ætérna damnatióne nos éripi, et in electórum tuórum iúbeas grege numerári. Per Christum, Dóminum nostrum. Amen.}
}

\ParallelRText{
\textcolor{my-color}{Prosimy Cię przeto, Panie, abyś łaskawie przyjął tę ofiarę od nas sług Twoich, jak również od całego ludu Twego,
a dni nasze raczył pokojem swym napełnić, od potępienia wiecznego nas uchronić i do grona wybranych swoich zaliczyć. Przez Chrystusa, Pana naszego. Amen.}
}
\end{Parallel}

\begin{center}
\textbf{Prośba o przeistoczenie}
\end{center}

\begin{Parallel}[v]{0.485\textwidth}{0.485\textwidth}
\ParallelLText{
\textcolor{my-color}{Quam oblatiónem tu, Deus, in ómnibus, quaesumus, bene\SmallCross díctam, adscríp\SmallCross tam, ra\SmallCross tam, rationábilem, acceptabilémque fácere dignéris: ut nobis Co \SmallCross pus, et San\SmallCross guis fiat dilectíssimi Fílii tui, Dómini nostri Iesu Christi. }
}

\ParallelRText{
\textcolor{my-color}{Racz te dary ofiarne, prosimy Cię, Boże, w całej pełni pobłogosławić, przyjąć, zatwierdzić, uduchowić i miłymi sobie uczynić, aby się nam stały Ciałem i Krwią najmilszego Syna Twego, Pana naszego Jezusa Chrystusa.}
}
\end{Parallel}

\begin{center}
\textbf{Konsekracja chleba}
\end{center}

\begin{center}
\textcolor{red}{Kapłan bierze do rąk Hostię.}
\end{center}

\begin{Parallel}[v]{0.485\textwidth}{0.485\textwidth}
\ParallelLText{
\textcolor{my-color}{Qui prídie quam paterétur, accépit panem in sanctas ac venerábiles manus suas, elevátis óculis in coelum ad te Deum, Patrem suum omnipoténtem, tibi grátias agens, bene\SmallCross dixit, fregit, dedítque discípulis suis, dicens: Accípite, et manducáte ex hoc omnes. }
}

\ParallelRText{
\textcolor{my-color}{On to w przeddzień męki wziął chleb w swoje święte i czcigodne ręce, a podniósłszy oczy w niebo ku Tobie, Bogu, Ojcu swemu wszechmogącemu, dzięki Ci składając pobłogosławił, połamał i rozdał uczniom swoim mówiąc: Bierzcie i jedzcie z tego wszyscy:}
}
\end{Parallel}

\begin{center}
\textcolor{red}{Kapłan pochyla się nad Hostią.}
\end{center}

\begin{Parallel}[v]{0.485\textwidth}{0.485\textwidth}
\ParallelLText{
\textcolor{my-color}{HOC EST ENIM CORPUS MEUM.}
}
\ParallelRText{
\textcolor{my-color}{TO JEST BOWIEM CIAŁO MOJE.}
}
\end{Parallel}

\begin{center}
\textcolor{red}{Gdy kapłan podnosi Hostię ministrant dzwoni, a wierni adorują w ciszy.}
\end{center}

\begin{center}
\textbf{Konsekracja wina}
\end{center}

\begin{center}
\textcolor{red}{Kapłan bierze do rąk kielich.}
\end{center}

\begin{Parallel}[v]{0.485\textwidth}{0.485\textwidth}
\ParallelLText{
\textcolor{my-color}{Símili modo postquam coenátum est, accípiens et hunc præclárum Cálicem in sanctas ac venerábiles manus suas: item tibi grátias agens, bene\SmallCross dixit, dedítque discípulis suis, dicens: Accípite, et bíbite ex eo omnes.}
}

\ParallelRText{
\textcolor{my-color}{Podobnie po wieczerzy wziął i ten kielich wspaniały w swoje święte i czcigodne ręce, a ponownie dzięki Ci składając, pobłogosławił i podał uczniom swoim, mówiąc: Bierzcie i pijcie z niego wszyscy:}
}
\end{Parallel}

\begin{center}
\textcolor{red}{Kapłan pochyla się nad kielichem.}
\end{center}

\begin{Parallel}[v]{0.485\textwidth}{0.485\textwidth}
\ParallelLText{
\textcolor{my-color}{HIC EST ENIM CALIX SANGUINIS MEI, NOVI ET AETERNI TESTAMENTI: MYSTERIUM FIDEI: QUI PRO VOBIS ET PRO MULTIS EFFUNDETUR IN REMISSIONEM PECCATORUM.}

\textcolor{my-color}{Hæc quotiescúmque fecéritis, in mei memóriam faciétis.}
}

\ParallelRText{
\textcolor{my-color}{TO JEST BOWIEM KIELICH KRWI MOJEJ, NOWEGO I WIECZNEGO PRZYMIERZA: TAJEMNICA WIARY: KTÓRA BĘDZIE WYLANA ZA WAS I ZA WIELU NA ODPUSZCZENIE GRZECHÓW.}

\textcolor{my-color}{Ilekroć to czynić będziecie, na moją pamiątkę czyńcie.}
}
\end{Parallel}

\begin{center}
\textcolor{red}{Gdy kapłan podnosi Kielich ministrant dzwoni, a wierni adorują w ciszy.}
\end{center}

\begin{center}
\textbf{Wspomnienie tajemnicy odkupienia}
\end{center}

\begin{Parallel}[v]{0.485\textwidth}{0.485\textwidth}
\ParallelLText{

\textcolor{my-color}{Unde et mémores, Dómine, nos servi tui, sed et plebs tua sancta, eiusdem Christi Fílii tui, Dómini nostri, tam beátæ passiónis, nec non et ab ínferis resurrectiónis, sed et in coelos gloriósæ ascensiónis: offérimus præcláræ maiestáti tuæ de tuis donis ac datis, hóstiam\SmallCross puram, hóstiam\SmallCross sanctam, hóstiam\SmallCross immaculátam, Panem\SmallCross sanctum vitæ ætérnæ, et Calicem\SmallCross salútis perpétuæ.}
}

\ParallelRText{
\textcolor{my-color}{My przeto, Panie słudzy Twoi oraz lud Twój święty, pomni na błogosławioną Mękę i Zmartwychwstanie z otchłani, jak również na chwalebne Wniebowstąpienie tegoż Chrystusa Syna Twego, Pana naszego, składamy chwalebnemu majestatowi Twemu z otrzymanych od Ciebie darów ofiarę czystą, ofiarę świętą, ofiarę niepokalaną, Chleb święty żywota wiecznego i Kielich wiekuistego zbawienia.}
}
\end{Parallel}

\begin{center}
\textbf{Moditwa o przyjęcie ofiary}
\end{center}

\begin{Parallel}[v]{0.485\textwidth}{0.485\textwidth}
\ParallelLText{
\textcolor{my-color}{Supra quæ propítio ac seréno vultu respícere dignéris: et accépta habére, sicúti accépta habére dignátus es múnera púeri tui iusti Abel, et sacrifícium Patriárchæ nostri Abrahæ: et quod tibi óbtulit summus sacérdos tuus Melchísedech, sanctum sacrifícium, immaculátam hóstiam.}
}

\ParallelRText{
\textcolor{my-color}{Racz wejrzeć na nie miłościwym i pogodnym obliczem i z upodobaniem przyjąć, jak raczyłeś przyjąć dary sługi Swego sprawiedliwego Abla i ofiarę Patriarchy naszego Abrahama, oraz tę, którą Ci złożył najwyższy Twój kapłan Melchizedek, ofiarę świętą, hostię niepokalaną.}
}
\end{Parallel}

\begin{center}
\textbf{Druga modlitwa wstawiennicza}
\end{center}

\begin{Parallel}[v]{0.485\textwidth}{0.485\textwidth}
\ParallelLText{
\textcolor{my-color}{Súpplices te rogámus, omnípotens Deus: iube hæc perférri per manus sancti Angeli tui in sublíme altáre tuum, in conspéctu divínæ maiestátis tuæ: ut, quotquot ex hac altáris participatióne sacrosánctum Fílii tui Cor\SmallCross pus, et Sán\SmallCross guinem sumpsérimus, omni benedictióne coelésti et grátia repleámur. Per eúndem Christum, Dóminum nostrum. Amen.}

\textcolor{my-color}{Meménto étiam, Dómine, famulórum famularúmque tuárum N. et N., qui nos præcessérunt cum signo fídei, et dórmiunt in somno pacis. Ipsis, Dómine, et ómnibus in Christo quiescéntibus locum refrigérii, lucis pacis ut indúlgeas, deprecámur. Per eúndem Christum, Dóminum nostrum. Amen.}
}

\ParallelRText{
\textcolor{my-color}{Pokornie Cię błagamy, wszechmogący Boże, rozkaż, niech ręce Twego Anioła świętego zaniosę tę ofiarę na niebieski Twój ołtarz, przed oblicze Boskiego majestatu Twego, abyśmy wszyscy, gdy jako uczestnicy tej ofiary ołtarza przyjmować będziemy najświętsze Ciało i Krew Syna Twego, otrzymali z nieba pełnię błogosławieństwa i łaski. Przez tegoż Chrystusa, Pana naszego. Amen.}

\textcolor{my-color}{Pomnij też, Panie, na sługi i służebnice Twoje N. N., którzy nas poprzedzili ze znamieniem wiary i śpią snem pokoju. Im oraz wszystkim spoczywającym w Chrystusie użycz, błagamy Cię, Panie, miejsca ochłody, światłości i pokoju. Przez tegoż Chrystusa, Pana naszego. Amen.}
}
\end{Parallel}

\begin{center}
\textcolor{red}{Kapłan bije się w piersi.}
\end{center}

\begin{Parallel}[v]{0.485\textwidth}{0.485\textwidth}
\ParallelLText{
\textcolor{my-color}{Nobis quoque peccatóribus fámulis tuis, de multitúdine miseratiónum tuárum sperántibus, partem áliquam et societátem donáre dignéris, cum tuis sanctis Apóstolis et Martýribus: cum Ioánne, Stéphano, Matthía, Bárnaba, Ignátio, Alexándro, Marcellíno, Petro, Felicitáte, Perpétua, Agatha, Lúcia, Agnéte, Cæcília, Anastásia, et ómnibus Sanctis tuis: intra quorum nos consórtium, non æstimátor mériti, sed véniæ, quæsumus, largítor admítte. Per Christum, Dóminum nostrum. }

\textcolor{my-color}{Per quem hæc ómnia, Dómine, semper bona creas, sancti\SmallCross ficas, viví\SmallCross ficas, bene\SmallCross dícis et præstas nobis.
Per ip\SmallCross sum, et cum ip\SmallCross so, et in ip\SmallCross so, est tibi Deo Patri\SmallCross omnipotenti, in unitáte Spíritus\SmallCross Sancti,
omnis honor, et glória.}

Per omnia sæcula sæcolorum.

\response Amen.
}

\ParallelRText{
\textcolor{my-color}{Nam również, grzesznym sługom Twoim, którzy pokładamy nadzieję w ogromie miłosierdzia Twego, racz przyznać jakąś cząstkę i wspólnotę ze świętymi Apostołami i Męczennikami Twoimi: Janem, Szczepanem, Maciejem, Barnabą, Ignacym, Aleksandrem, Marcelinem, Piotrem, Felicytą, Perpetuą, Agatą, Łucją, Agnieszką, Cecylią, Anastazją i wszystkimi Świętymi Twoimi; prosimy Cię, dopuść nas do ich grona nie jako sędzia zasługi, lecz jako dawca przebaczenia. Przez Chrystusa, Pana naszego.}

\textcolor{my-color}{Przez Niego, Panie, wszystkie te dobra ustawicznie stwarzasz, uświęcasz, ożywiasz, błogosławisz i nam ich udzielasz.
Przez Niego i z Nim, i w Nim masz, Boże Ojcze wszechmogący, w jedności Ducha Świętego, wszelką cześć i chwałę.}

Przez wszystkie wieki wieków.

\response Amen.
}
\end{Parallel}

\begin{center}
\textcolor{red}{\textbf{Wierni wstają.}}
\end{center}

\begin{center}
\textbf{Moditwa Pańska}
\end{center}

\begin{center}
\textcolor{red}{Teraz po złożeniu Ofiary, następuje uczta. W odpowiedzi na nasze dary, prośby i błagania Bóg udzieli nam pokarmu świętego, w którym są zawarte wszystkie łaski i dobrodziejstwa, bo pokarm ten, to sam Pan Jezus pod postaciami chleba i wina. Jest to stwierdzenie, że Bóg nas wysłuchał i chce nam dopomóc, jako swym dzieciom. Jako przygotowanie do Komunii św. odmówmy z głębi serca "Ojcze nasz" i prośmy Pana Jezusa, aby to Jego Ciało, które spożywać będziemy, stało się nam lekarstwem dla duszy i ciała. Wierni nie śpiewają z kapłanem „Pater Noster”, śpiewają dopiero ostatnią prośbę „sed libera nos a malo”.}
\end{center}

\begin{Parallel}[v]{0.485\textwidth}{0.485\textwidth}
\ParallelLText{
Orémus: Præcéptis salutáribus móniti, et divína institutione formati audemus dicere:

Pater noster, qui es in caelis, Sanctificetur nomen tuum. Adveniat regnum tuum. Fiat voluntas tua, sicut in coelo et in terra. Panem nostrum quotidianum da nobis hodie. Et dimitte nobis debita nostra, sicut et nos dimittimus debitoribus nostris. Et ne nos inducas in tentationem:

\response Sed libera nos a malo.

\textcolor{red}{S.} Amen.
}

\ParallelRText{
Módlmy się. Wezwani zbawiennym nakazem i oświeceni pouczeniem Bożym ośmielamy się mówić:

Ojcze nasz, któryś jest w niebie: święć się Imię Twoje, przyjdź królestwo Twoje, bądź wola Twoja jako w niebie tak i na ziemi. Chleba naszego powszedniego daj nam dzisiaj i odpuść nam nasze winy, jako i my odpuszczamy naszym winowajcom. I nie wódź nas na pokuszenie.

\response Ale nas zbaw ode złego.

\textcolor{red}{S.} Amen.
}
\end{Parallel}

\begin{center}
\textbf{Błogosławieństwo nowożeńców}
\end{center}

\begin{center}
\textcolor{red}{Nowo zaślubiona para klęka, ksiądz odwraca się do pary młodej i odmawia nad nowożeńcami modlitwy:}
\end{center}

\begin{Parallel}[v]{0.485\textwidth}{0.485\textwidth}
\ParallelLText{
Propitiare Domine supplicantionibus nostris, et institutis tuis, quibus propagationem humani generis ordinasti, benignus assiste: ut, quod te auctore iungitur, te auxilliante servetur. Per Dominum nostrum…

Deus, qui potestate virtutis tuae de nihilo cuncta fecisti: qui dispositis miversitatis exordiis, homini ad imaginem Dei facto, ideo inseparabile mulieris adiutorium con didisti, ut femineo corpora de virili dares carne principium, docens, quod ex uno placuisset insttuí, numquam licere disiungi: Deus, qui tam excellenti mysterio coniugalem copulam consecrasti, ut Christi et Ecclesiae sacramentum praesignares in foedere nuptiarum: Deus, per quem mulier iungitur viro, et societas principaliter ordinata, ea benedictione donatur, quae sola nec per originalis peccati poenam nec per diluvii est ablata sententiam: respice propitius super hanc famulam tuam, quae, maritali iungenda consortio, tua se ex petit protectione muniri: sit in ea iugum dilectionis et pacis: fidelis et casta nubat in Christo, imita trixque sanctarum permaneat feminarum: sit amabilis viro suo, ut Rachel: sapiens, ut Rebecca: longaeva et fidelis, ut Sara: nihil in ea ex actibus suis illie auctor praevaricationis usurpet: nexa fidei datisque permaneat unithoro iuncta, contacts illicitos fugiat: muniat infirmitatem suam robore disciplinae: sit verecundia gravis, pudore venerabilis doctrinis caelestibus erudita: sit fecunda in sobole, sit probata et innocens: et ad Beatorum requiem atque ad cáelestia regna perveniat: et videánt ambo filios filiorum suorum usque in tertiam et quartam generationem, et ad optatam pérveinirent senectutem. Per eundem Doiminum nostrum…
}

\ParallelRText{
Daj się przebłagać, Panie, naszymi modłami i błogosław dobrotliwie związkowi, który ustanowiłeś dla rozkrzewienia rodzaju ludzkiego, aby po myśli Twojej zawarty, w Twej łasce trwał nadal niewzruszenie. Przez Pana naszego…

Boże, Tys wszechpotężną mocą swoją wszystko z niczego stworzył; Ty po uporządkowaniu wszechświata, człowiekowi, na obraz Boży stworzonemu, dałeś niewiastę jako nieodłączną towarzyszkę życia i to w ten sposób, że ciało jej z ciała mężczyzny utworzyłeś, by naocznie okazać, że tego, co Ci się spodobało z jednego uczynić tworzywa, nigdy rozłączać niewolno, Boże, Tyś związek małżeński przez tak podniosłą tajemnicę uświęcił, chciałeś bowiem, by był on obrazem pełnego tajemnic związku Chrystusa z Kościołem. Boże, z Twojej to woli niewiasta łączy się z mężem, który to związek od początku świata ustanowiony takim błogosławieństwem uświącasz, iż go ani wina i kara grzechu pierworodnego, ani wyrok karzącego potopu zniweczyć nie zdołały. Wejrzyj tedy laskawie na tę służebnicę Twoją, która w związek małżeński wstępując, Twojej uprasza i Twojej oczekuje opieki. Niech znajdzie w tym związku słodkie jarzmo miłości i pokoju. Niechaj to życie małżeńskie rozpocznie w Chrystusie jako małżonka wierna i czysta. Niech trwa w naśladowaniu świętych niewiast, niech będzie miła swemu mężowi jak Rachel, roztropna jak Rebeka, niech żyje długo i niech będzie wierna jak Sara. Niech sprawca wszelkiej przewrotności (szatan) ani jednego z jej czynów sobie nie przypisze. Wytrwała w wierze i w pełnieniu przykazań Twoich niech z drogi tej nigdy nie schodzi. Jednemu poślubiona mężowi, niech niedozwolonych stosunków unika. Niech wzmacnia swą słabość siłą karności, niech potęguje swą godność przez skromność, a wstydliwość niechaj czyni ją czcigodną. W nauce świętej niech będzie biegłą, niech się doczeka licznego potomstwa, niech będzie prawą i wolną od winy, a kiedyś niech dostąpi pokoju z błogosławionymi i dojdzie do Królestwa niebieskiego. Oboje zaś niech oglądają dzieci swoje aż do trzeciego i czwartego pokolenia, niech osięgną szczęśliwy wiek sędziwy. Przez tegoż Pana naszego, Jezusa Chrystusa, Syna Twego, który z Tobą żyje i króluje w jedności z Duchem Świętym, I Bóg po wszystkie wieki. Amen.
}
\end{Parallel}

\begin{center}
\textcolor{red}{Kapłan rozwija ostatnią prośbę.}
\end{center}

\begin{Parallel}[v]{0.485\textwidth}{0.485\textwidth}
\ParallelLText{
Libera nos, quaesumus, Domine ab omnibus malis, praeteris, praesentiset futuris: et intercedente beata et gloriosa semper Virgine Dei Genitrice Maria, cum beatis Apostolis tui Petro et Paulo, atque Andrea, et omnibus Sanctis, da propitius pacem in diebus nostris: ut, ope misericordiae tuae adjuti, et a peccato simus semper liberi et ab omni perturbatione securi. Per eumden Dominum nostrum Jesum Christum, Filium tuum. Qui tecum vivit et regnat in unitate Spiritus Sancti Deus.
}

\ParallelRText{
Wybaw nas, prosimy Cię, Panie od wszelkich nieszczęść przeszłych, obecnych i przyszłych, a za przyczyną Najświętszej i chwalebnej zawsze Dziewicy, Bogarodzicy Maryi, świętych Apostołów Twoich Piotra i Pawła, oraz Andrzeja i wszystkich Świętych, udziel nam miłościwie pokoju za dni naszych, abyśmy wsparci pomocą miłosierdzia Twego i od grzechu byli zawsze wolni i od wszelkiej trwogi bezpieczni. Przez tegoż Pana naszego Jezusa Chrystusa, Syna Twojego, który z Tobą żyje i króluje w jedności Ducha Świętego Bóg.
}
\end{Parallel}

\begin{center}
\textbf{Łamanie Chleba i modły o pokój}
\end{center}

\begin{center}
\textcolor{red}{Obrzęd Łamania Chleba jest również symbolem gwałtownej śmierci Chrystusa, połączenie zaś Hostii z Krwią wskazuje na Jego zmartwychwstanie. Pierwszym owocem Eucharystii jest pokój, czyli jedność z Bogiem i jedność między nami.}
\end{center}

\begin{Parallel}[v]{0.485\textwidth}{0.485\textwidth}
\ParallelLText{
\versicle Per omnia saecula saeculorum.

\response Amen.

Pax Domini sit semper vobiscum.\\

\response Et cum spiritu tuo.
}

\ParallelRText{
\versicle Przez wszystkie wieki wieków.

\response Amen.

Pokój Pański niech zawsze będzie z wami.

\response I z duchem twoim.
}
\end{Parallel}

\begin{center}
\textcolor{red}{Kapłan wpuszcza cząstkę Hostii Św. do Kielicha, mówiąc:}
\end{center}

\begin{Parallel}[v]{0.485\textwidth}{0.485\textwidth}
\ParallelLText{
Haec commíxtio, et consecrátio Córporis et Sánguinis Dómini nostri Iesu Christi, fiat accipiéntibus nobis in vitam ætérnam. Amen.
}

\ParallelRText{
To sakramentalne połączenie Ciała i Krwi Pana naszego Jezusa Chrystusa, którego mamy przyjąć, niech się nam przyczyni do żywota wiecznego. Amen.
}
\end{Parallel}

\begin{center}
\textcolor{red}{Kapłan wpuszcza cząstkę Hostii Św. do Kielicha, mówi, a chór śpiewa, \textbf{wierni klękają}}
\end{center}

\begin{Parallel}[v]{0.485\textwidth}{0.485\textwidth}
\ParallelLText{
Agnus Dei, qui tollis peccáta mundi: miserére nobis.

Agnus Dei, qui tollis peccáta mundi: miserére nobis.

Agnus Dei, qui tollis peccáta mundi: dona nobis pacem.
}

\ParallelRText{
Baranku Boży, który gładzisz grzechy świata, zmiłuj się nad nami.

Baranku Boży, który gładzisz grzechy świata, zmiłuj się nad nami.

Baranku Boży, który gładzisz grzechy świata, obdarz nas pokojem.
}
\end{Parallel}

\begin{center}
\textbf{Modlitwy przed Komunią}
\end{center}

\begin{center}
\textcolor{red}{Następują trzy ciche prywatne modlitwy kapłana, pięknie przypominające skutki, jakie Komunia ma spowodować w duszach naszych: pokój, uzdrowienie, łaskę Bożą. Opieramy się w tej chwili na zasługach Chrystusa i wierze Kościoła.}
\end{center}

\begin{Parallel}[v]{0.485\textwidth}{0.485\textwidth}
\ParallelLText{
\textcolor{my-color}{Dómine Iesu Christe, qui dixísti Apóstolis tuis: Pacem relínquo vobis, pacem meam do vobis: ne respícias peccáta mea, sed fidem Ecclésiæ tuæ; eámque secúndum voluntátem tuam pacificáre et coadunáre dignéris: Qui vivis et regnas Deus per ómnia saecula sæculórum. Amen.}

\textcolor{my-color}{Dómine Iesu Christe, Fili Dei vivi, qui ex voluntáte Patris, cooperánte Spíritu Sancto, per mortem tuam mundum vivificásti: líbera me per hoc sacrosánctum Corpus et Sánguinem tuum ab ómnibus iniquitátibus meis, et univérsis malis: et fac me tuis semper inhærére mandátis, et a te numquam separári permíttas: Qui cum eódem Deo Patre et Spíritu Sancto vivis et regnas Deus in saecula sæculórum. Amen.}

\textcolor{my-color}{Percéptio Córporis tui, Dómine Iesu Christe, quod ego indígnus súmere præsúmo, non mihi provéniat in iudícium et condemnatiónem: sed pro tua pietáte prosit mihi ad tutaméntum mentis et córporis, et ad medélam percipiéndam: Qui vivis et regnas cum Deo Patre in unitáte Spíritus Sancti Deus, per ómnia saecula sæculórum. Amen.}
}

\ParallelRText{
\textcolor{my-color}{Panie Jezu Chryste, któryś rzekł Apostołom Swoim: Pokój zostawiam wam, pokój Mój wam daję, nie zważaj na grzechy moje, lecz na wiarę Kościoła Swego i według woli Swojej racz go darzyć pokojem i utwierdzać w jedności: Który żyjesz i królujesz, Bóg przez wszystkie wieki wieków. Amen.}

\textcolor{my-color}{Panie Jezu Chryste, Synu Boga żywego, który z woli Ojca, za sprawą Ducha Świętego przez śmierć Swoją dałeś życie światu, wyzwól mię przez to najświętsze Ciało i Krew Swoją od wszystkich nieprawości moich i od wszelkiego zła; spraw także, bym zawsze lgnął do przykazań Twoich i nie dozwól mi nigdy odłączyć się od Ciebie: Który z tymże Bogiem Ojcem i Duchem Świętym żyjesz i królujesz, Bóg na wszystkie wieki wieków. Amen.}

\textcolor{my-color}{Panie Jezu Chryste, przyjęcie Ciała Twego, które ja niegodny ośmielam się spożyć, niech mi nie wyjdzie na sąd i potępienie, ale z miłościwej dobroci Twojej niech będzie dla mnie ochroną duszy i ciała oraz skutecznym lekarstwem: Który żyjesz i królujesz z Bogiem Ojcem w jedności Ducha Świętego Bóg, przez wszystkie wieki wieków. Amen.}
}
\end{Parallel}

\begin{center}
\textbf{Komunia kapłana}
\end{center}

\begin{center}
\textcolor{red}{Kapłan przyklęka, bierze do rąk Hostię św., aby ją przyjąć i mówi:}
\end{center}

\begin{Parallel}[v]{0.485\textwidth}{0.485\textwidth}
\ParallelLText{
\textcolor{my-color}{Panem coelestem accipiam et nomen Domini invocabo.}
}

\ParallelRText{
\textcolor{my-color}{Chleb niebiański przyjmę i wezwę Imienia Pana.}
}
\end{Parallel}

\begin{center}
\textcolor{red}{Trzymając Hostię Świętą w lewej dłoni, kapłan uderza się w piersi trzy razy.}
\end{center}

\begin{Parallel}[v]{0.485\textwidth}{0.485\textwidth}
\ParallelLText{
\textcolor{my-color}{Domine, non sum dignus ut intres sub tectum meum: sed tantum dic verbo, et sanabitur anima mea (ter.)}
}

\ParallelRText{
\textcolor{my-color}{Panie, nie jestem godzien, abyś wszedł do wnętrza mego, ale rzeknij tylko słowem, a będzie uzdrowiona dusza moja. (trzykrotnie)}
}
\end{Parallel}

\begin{center}
\textcolor{red}{Trzymając Hostię Świętą w prawej ręce, kapłan czyni Nią znak krzyża i mówi:}
\end{center}

\begin{Parallel}[v]{0.485\textwidth}{0.485\textwidth}
\ParallelLText{
\textcolor{my-color}{Corpus Domini nostri Jesu Christi custodiat animam meam in vitam aeternam. Amen.}
}

\ParallelRText{
\textcolor{my-color}{Ciało Pana naszego Jezusa Chrystusa niechaj strzeże duszy mojej na żywot wieczny. Amen.}
}
\end{Parallel}

\begin{center}
\textcolor{red}{Po chwili kapłan bierze do rąk Kielich i modli się słowami psalmu, który Chrystus odmawiał w czasie Ostatniej Wieczerzy:}
\end{center}

\begin{Parallel}[v]{0.485\textwidth}{0.485\textwidth}
\ParallelLText{
\textcolor{my-color}{Quid retribuam Domino pro omnibus, quae retribuit mihi? Calicem salutaris accipiam, et nomen Domini invocabo. Laudans invocabo Dominum, et ab inimicis meis salvus ero.}

\textcolor{my-color}{Sanguis Domini nostri Jesu Christi custodiat animam meam in vitam aeternam. Amen.}
}

\ParallelRText{
\textcolor{my-color}{Cóż zwrócę Panu za wszystko, co dla mnie uczynił; Kielich zbawienia podnosę i wezwę Imienia Pana. Wielbiąc zawołam do Pana i od nieprzyjaciół moich będę ocalony.}

\textcolor{my-color}{Krew Pana naszego Jezusa Chrystusa niechaj strzeże duszy mojej na żywot wieczny. Amen.}
}
\end{Parallel}

\begin{center}
\textbf{Komunia wiernych}
\end{center}

\begin{center}
\textcolor{red}{Kapłan zwraca się do wiernych z Cyborium i trzymając Hostię Świętą, mówi:}
\end{center}

\begin{Parallel}[v]{0.485\textwidth}{0.485\textwidth}
\ParallelLText{
Ecce Agnus Dei, ecce Qui tollit peccata mundi.
}

\ParallelRText{
Oto Baranek Boży: oto który gładzi grzechy świata
}
\end{Parallel}

\begin{center}
\textcolor{red}{Wierni biją się w piersi i mówią wraz z ministrantami (kapłanem):}
\end{center}

\begin{Parallel}[v]{0.485\textwidth}{0.485\textwidth}
\ParallelLText{
Domine, non sum dignus ut intres sub tectum meum: sed tantum dic verbo, et sanabitur anima mea (ter.)
}

\ParallelRText{
Panie, nie jestem godzien, abyś wszedł do wnętrza mego, ale rzeknij tylko słowem, a będzie uzdrowiona dusza moja. (trzykrotnie)
}
\end{Parallel}

\begin{center}
\textcolor{red}{Komunię przyjmuje się w postawie klęczącej, wyrażając szacunek dla Boga, który realnie obecny jest w każdej cząstce Swojego Ciała, które będziemy spożywać. Postawa stojąca dopuszczalna jest tylko w wypadku, gdy jakaś osoba nie może uklęknąć ze względnu na stan zdrowia czy np. sędziwy wiek. Podając Komunię Świętą, kapłan mówi:}
\end{center}

\begin{Parallel}[v]{0.485\textwidth}{0.485\textwidth}
\ParallelLText{
Corpus Domini nostri Jesu Christi custodiat animam tuam in vitam aeternam. Amen.
}

\ParallelRText{
Ciało Pana naszego Jezusa Chrystusa niechaj strzeże duszy Twojej na żywot wieczny. Amen.
}
\end{Parallel}

\begin{center}
\textcolor{red}{Wierny nie odpowiada „amen”, w ogóle nic nie mówi. Gdy wszyscy otrzymają Komunię Świętą, kapłan powraca do ołtarza i umieszcza Cyborium w Tabernakulum.}
\end{center}

\begin{center}
\textbf{Dziękczynienie}
\end{center}

\begin{center}
\textcolor{red}{Po czym ministrant bierze ampułki i nalewa wina do kielicha. Kapłan modli się:}
\end{center}

\begin{Parallel}[v]{0.485\textwidth}{0.485\textwidth}
\ParallelLText{
\textcolor{my-color}{Quod ore sumpsimus Domine, pura mente capiamus: et de munere temporali fiat nobis remedium sempiternum.}
}

\ParallelRText{
\textcolor{my-color}{Cośmy usty spożyli, Panie, daj czystym przyjąć umysłem, a dar ten doczesny niech się nam stanie lekarstwem na wieczność.}
}
\end{Parallel}

\begin{center}
\textcolor{red}{Po czym ministrant bierze ampułki i nalewa wina do kielicha. Kapłan modli się:}
\end{center}

\begin{Parallel}[v]{0.485\textwidth}{0.485\textwidth}
\ParallelLText{
\textcolor{my-color}{Corpus tuum, Domine, quod sumpsi, et Sanguis, quem potavi, adhaereat visceribus meis: et praesta, ut in me non remaneat scelerum macula, quem pura et sancta refecerunt sacramenta. Qui vivis et regnas in saecula saeculorum. Amen.}
}

\ParallelRText{
\textcolor{my-color}{Ciało Twe, Panie, które spożyłem, i Krew, którą wypiłem, niech przywrze do mego wnętrza, i spraw, aby zmaza grzechów nie została we mnie, którego czyste i święte posiliły Sakramenta. Który żyjesz i królujesz na wieki wieków. Amen.}
}
\end{Parallel}

\begin{center}
\textcolor{red}{Kapłan wyciera kielich i przykrywa go welonem - ministrant przenosi Mszał.}
\end{center}

\begin{center}
\textbf{Communio --- Śpiew przy Komunii}
\end{center}

\begin{center}
\textcolor{red}{Po czym kapłan przechodzi na stronę Lekcji i odczytuje Komunię.}
\end{center}

\begin{Parallel}[v]{0.485\textwidth}{0.485\textwidth}
\ParallelLText{
\textcolor{my-color}{Ecce, sic benedicetur omnis homo, qui timet Dominum: et videas filios filiorum quorum: pax super Israel.}
}

\ParallelRText{
\textcolor{my-color}{Oto tak pobłogosławiony będzie człowiek, który Pana się boi. Oglądać będziesz dzieci synów swoich. Pokój nad Izraelem.}
}
\end{Parallel}

\begin{center}
\textbf{Postcommunio --- Modlitwy po Komunii}
\end{center}

\begin{Parallel}[v]{0.485\textwidth}{0.485\textwidth}
\ParallelLText{
\versicle Dominus vobiscum.

\response Et cum spiritu tuo. 

Oremus.

Queasumus, omnipotens Deus: instituta providentiae tuae pio favore comitare; ut, quos legitima societate connectis, longaeva pace custodias. Per Dominum nostrum\ldots \\ \\

Per omnia saecula saeculorum.

\response Amen.
}

\ParallelRText{
\versicle Pan z wami.

\response. I z duchem Twoim. 

Módlmy się.

Prosimy Cię, wszechmogący Boże, niech twa ojcowska łaska towarzyszy bezustannie związkowi przez Ciebie ustanowionemu, aby ci których prawowicie złączyłeś, pod Twoją opieką długoletnim cieszyli się pokojem. Przez Pana naszego\ldots

Przez wszystkie wieki wieków.

\response Amen.
}
\end{Parallel}

\begin{center}
\textcolor{red}{Po skończonych modlitwach kapłan wraca na środek ołtarza, całuje go i zwracając się do wiernych, mówi:}
\end{center}

\begin{Parallel}[v]{0.485\textwidth}{0.485\textwidth}
\ParallelLText{
\versicle Dominus vobiscum.

\response Et cum spiritu tuo. 

\versicle Ite, missa est.

\response Deo gracias.
}

\ParallelRText{
\versicle Pan z wami.

\response I z duchem Twoim. 

\versicle Idźcie, Msza się skończyła.

\response Bogu dzięki.
}
\end{Parallel}

\begin{center}
\textcolor{red}{Po czym celebrans odmawia nad małżonkami następującą modlitwę:}
\end{center}

\begin{Parallel}[v]{0.485\textwidth}{0.485\textwidth}
\ParallelLText{
Deus Abraham, Deus Isaac et Deus Iacob sit vobiscum: et ipse adimpleat benedictionem suam in vobis: ut videatis filios filiorum vestrorum usque ad tertiam et quartam generationem, et postea vitam aeternam habeatis sine fine: adiuvante Domino nostro Iesu Christo, qui cum Patre et Spiritu Sancto vivit, et regnat Deus, per omnia saecula saeculorum. Amen.
}

\ParallelRText{
Bóg Abrahama, Bóg Izaaka i Bóg Jakuba niech będzie z wami. Niech zleje na was hojne swoje błogosławieństwo, abyście oglądali dzieci synów swoich aż do trzeciego i czwartego pokolenia, a wreszcie osiągnęli żywot wieczny z pomocą Pana naszego, Jezusa Chrystusa, który z Ojcem i Duchem Świętym żyje i króluje, Bóg na wieki wieków. Amen.
}
\end{Parallel}


\begin{center}
\textbf{Ostatnia modlitwa}
\end{center}

\begin{center}
\textcolor{red}{Wieni klękają. Pochylając się nad ołtarzem, kapłan mówi:}
\end{center}

\begin{Parallel}[v]{0.485\textwidth}{0.485\textwidth}
\ParallelLText{
\textcolor{my-color}{Pláceat tibi, sancta Trínitas, obséquium servitútis meæ: et præsta; ut sacrifícium, quod óculis tuæ maiestátis indígnus óbtuli, tibi sit acceptábile, mihíque et ómnibus, pro quibus illud óbtuli, sit, te miseránte, propitiábile. Per Christum, Dóminum nostrum. Amen.}
}

\ParallelRText{
\textcolor{my-color}{Trójco Przenajświętsza, przyjmij z upodobaniem hołd swego sługi i spraw, niech ta ofiara, którą ja niegodny złożyłem przed obliczem Twego majestatu, Tobie będzie miła, mnie zaś i wszystkim, za których ją ofiarowałem, niech przez miłosierdzie Twoje zjedna przebaczenie. Przez Chrystusa Pana naszego. Amen.}
}
\end{Parallel}


\begin{center}
\textbf{Błogosławieństwo}
\end{center}

\begin{center}
\textcolor{red}{Kapłan całuje ołtarz, a na słowo ,,Pater'' odwraca się do wiernych, \textbf{wierni klękają}}
\end{center}

\begin{Parallel}[v]{0.485\textwidth}{0.485\textwidth}
\ParallelLText{
Benedícat vos omnípotens Deus,

Pater, et Fílius,\SmallCross et Spíritus Sanctus.

\response Amen.
}

\ParallelRText{
\noindent Niech was błogosławi wszechmogący Bóg, Ojciec, Syn\SmallCross i Duch Święty.

\response Amen.
}
\end{Parallel}

\begin{center}
\textbf{Ostatnia Ewangelia}
\end{center}

\begin{center}
\textcolor{red}{\textbf{Wierni wstają.} Kapłan przechodzi na stronę Ewangelii. Czyni znak krzyża najpierw na ołtarzu, potem na swym czole, ustach i sercu.}
\end{center}

\begin{Parallel}[v]{0.485\textwidth}{0.485\textwidth}
\ParallelLText{

\versicle Dominus vobiscum.

\response Et cum spiritu tuo. 

\SmallCross Initium sancti Evangelii secundum Joannem.

\response Gloria tibi Domine.

\textcolor{my-color}{In princípio erat Verbum, et Verbum erat apud Deum, et Deus erat Verbum. Hoc erat in princípio apud Deum. Omnia per ipsum facta sunt: et sine ipso factum est nihil, quod factum est: in ipso vita erat, et vita erat lux hóminum: et lux in ténebris lucet, et ténebræ eam non comprehendérunt.
Fuit homo missus a Deo, cui nomen erat Ioánnes. Hic venit in testimónium, ut testimónium perhibéret de lúmine, ut omnes créderent per illum. Non erat ille lux, sed ut testimónium perhibéret de lúmine.
Erat lux vera, quæ illúminat omnem hóminem veniéntem in hunc mundum. In mundo erat, et mundus per ipsum factus est, et mundus eum non cognóvit. In própria venit, et sui eum non recepérunt. Quotquot autem recepérunt eum, dedit eis potestátem fílios Dei fíeri, his, qui credunt in nómine eius: qui non ex sanguínibus, neque ex voluntáte carnis, neque ex voluntáte viri, sed ex Deo nati sunt. Genuflectit dicens: Et Verbum caro factum est, Et surgens prosequitur: et habitávit in nobis: et vídimus glóriam eius, glóriam quasi Unigéniti a Patre, plenum grátiæ et veritatis.}

\response Deo gratias.
}

\ParallelRText{

\versicle Pan z wami.

\response I z duchem Twoim. 

\SmallCross Początek Ewangelii świętej według Jana.

\response Chwała Tobie Panie.

\textcolor{my-color}{Na początku było Słowo, a Słowo było u Boga i Bogiem było Słowo. Ono było na początku u Boga. Wszystko przez Nie się stało, a bez Niego nic się nie stało, co się stało. W Nim było życie, a życie było światłością ludzi, a światłość w ciemnościach świeci i ciemności jej nie ogarnęły.
Był człowiek posłany od Boga, a Jan mu było na imię. Przyszedł on na świadectwo, aby świadczyć o światłości, aby przez niego wszyscy uwierzyli. Nie był on światłością, ale miał świadectwo dać o światłości.
Była światłość prawdziwa, która oświeca każdego człowieka na ten świat przychodzącego. Na świecie był, a świat był przez Niego stworzony i świat Go nie poznał. Przyszedł do swojej własności, a swoi Go nie przyjęli. A wszystkim, którzy Go przyjęli i uwierzyli w imię Jego, dał moc, aby się stali synami Bożymi, którzy nie z krwi ani z żądzy ciała, ani też z woli ludzkiej, ale z Boga się narodzili.} \textcolor{red}{\textbf{(tu się przyklęka)}} \textcolor{my-color}{a Słowo stało się ciałem} \textcolor{red}{\textbf{(powstaje)}} \textcolor{my-color}{i mieszkało między nami, i widzieliśmy chwałę Jego, pełnego łaski i prawdy, chwałę jako Jednorodzonego od Ojca.}

\response Bogu dzięki.
}
\end{Parallel}


\begin{center}
\textbf{Wyjście asysty}
\end{center}

\begin{center}
\textcolor{red}{Śpiewa się pieśń:}
\end{center}

\begin{Parallel}[v]{0.485\textwidth}{0.485\textwidth}
\ParallelLText{
Salve Regina, Mater misericordiae, vita, dulcedo, et spes nostra, salve. Ad te clamamus, exsules filii Evae. Ad te suspiramus gementes et flentes in hac lacrymarum valle. Eia ergo, Advocata nostra, illos tuos misericordes oculos ad nos converte. Et Iesum, benedictum fructum ventris tui, nobis, post hoc exilium, ostende. O clemens, o pia, o dulcis Virgo Maria.
}

\ParallelRText{
Witaj Królowo, Matko Miłosierdzia, życie, słodyczy i nadziejo nasza, witaj! Do Ciebie wołamy wygnańcy, synowie Ewy, do Ciebie wzdychamy jęcząc i płacząc na tym łez padole. Przeto, Orędowniczko nasza, one miłosierne oczy Twoje na nas zwróć, a Jezusa, błogosławiony owoc żywota Twojego, po tym wygnaniu nam okaż. O łaskawa, o litościwa, o słodka Panno Maryjo!
}
\end{Parallel}

\begin{center}
	\tikzset{
		pgfornamentstyle/.style={scale=.25}
	}
	\foreach \i in {60} {\expandafter\pgfornament\expandafter{\i}\ }
	
\end{center}
